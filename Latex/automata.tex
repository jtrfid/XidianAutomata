\chapter{automata abstract}

\section{Finite automata}
\begin{definition}[Finite automation] 
	A finite automaton(an FA) is a 6-tuple $(Q,V,T,E,S,F)$ where
	\begin{itemize}
		\item $Q$ is a finite set of states,
		\item $V$ is an alphabet,
		\item $T\in \mathbb{P}(Q\times V\times Q)$ is a transition relation,
		\item $E\in\mathbb{P}(Q\times Q)$ is an $\epsilon$-transition relation
		\item $S\subseteq Q$ is a set of start states, and
		\item $F\subseteq Q$ is a set of final states.
	\end{itemize}   $\hfill\square$
\end{definition}

\begin{lstlisting}
class FA:virtual public FAabs {
	// Q is a finite set of states
	StatePool Q;    
	// S is a set of start states, F is a set of final states
	StateSet S, F;  
	// Transitions maps each State to its out-transitions.
	TransRel Transitions;
	// E is the epsilon transition relation.
	StateRel E;
}
\end{lstlisting}

\textbf{StatePool}:All states in an automaton are allocated from a StatePool. StatePool's can be merged together to form a larger one. (Care must be taken to rename any relations or sets (during merging) that depend on the one StatePool.)
State is in [0,next)
\begin{lstlisting}
class StatePool {
	int next; // The next one to be allocated.
}
\end{lstlisting}

\textbf{StateSet}:The StateSet is normally associated (implicitly) with a particular StatePool; whenever a StuteSet is to interact with another (from a different StatePool), its States must be renamed (to avoid a name clash). The capacity of a StateSet must be explicitly managed; many set operations are not bounds-checked when assert() is turned off.
\begin{lstlisting}
class StateSet :protected BitVec {
	// How many States can this set contain?
	// [O, domain()) can be contained in *this.
	inline int domain() const;
	
	// set How many States can this set contain.
	// [O, r) can be contained in *this.
	inline void set_domain(const int r);
}
class BitVec {
	// uesd max number bits in data, denote width(domain),[0,bits_in_use) == > [0,width)
	int bits_in_use; 
	// number of words,1,2,3,...
	int words; 
	// save bytes of words,[0,1,2,...width(domain)]
	unsigned int *data;
}
\end{lstlisting}

\textbf{transition relation:} $T \in Q \to \mathbb{P}(V \times Q),T(p)=\{(a,q)|(p,a,q)\in T\}$,表示状态$p$的out-transitions. see Fig \ref{fig:outT} 
\begin{lstlisting}
// V --> Q
struct TransPair {
CharRange transition_label;
State transition_destination;
}
class TransImpl { TransPair *data; }
class Trans:protected TransIml { }

// map: state(r) --> (T=Trans) out-transitions of r  
// SteteTo::data[r] = out-transitions of state r
class TransRel:public StateTo<Trans> {}

// map: state(r) --> T
// data[r] = T
template <class T> class SteteTo {
T *data; // 动态数组的index(即状态的index)状态的out-transitions 
}
class FA:virtual public FAabs {
TransRel Transitions;// maps each State to its out-transitions.
}
\end{lstlisting}

\textbf{$\epsilon$-relation}: $E\in \mathbb{P}(Q\times  Q) \Rightarrow E\in Q \to \mathbb{P}(Q),E(p)=\{q|(p,q)\in E\}$,表示$\epsilon$连接状态$p$和状态$q$.
\begin{lstlisting}
// Implement binary relations on States. This is most often used for epsilon transitions.
// map: state(r) --> {StateSet} 
// StateTo::data[r] = {StateSet}, 表示状态r与{StateSet}的二元关系
class StateRel :protected StateTo<StateSet> {
}

class FA:virtual public FAabs {
// E is the epsilon transition relation.
StateRel E;
}
\end{lstlisting}

\begin{figure}[htbp]
	\begin{tikzpicture}[>=latex, shorten >=1pt,
	node distance=0.75in, on grid, auto]
	\node[state, initial] (q0) {$q_0$};
	\node[state](q1) [right=of q0] {$q_1$};
	\node[state,accepting] (q2) [right=of q1] {$q_2$};
	\node[state] (q3) [below right=of q1] {$q_3$};
	\path[->] 
	(q0) edge node {$a$} (q1)
	(q1) edge node {$b$} (q2)
	(q1) edge [bend right] node {$c$} (q3)
	;
	\end{tikzpicture}
	\caption{$q_1$ in-transition: $\{(q_0,a,q_1)\}$;
	$q_1$ out-transition: $\{(q_1,b,q_2),(q_1,c,q_3)\}$}
    \label{fig:outT}
\end{figure}

\cite[p6]{WATSON93a} $\textbf{the signatures of the transition relations:}$

$T \in \mathbb{P}(Q \times V \times Q)$

$T \in V \to \mathbb{P}(Q \times Q)$

$T \in Q\times Q \to \mathbb{P}(V)$

$T \in Q \times V \to \mathbb{P}(Q)$

$T \in Q \to \mathbb{P}(V \times Q)$

for example, the function $T \in Q \to \mathbb{P}(V \times Q)$ is defined as $T(p) = \{(a,q):(p,a,q) \in T\}$


\hfill

$T \in \mathbb{P}(Q \times V \times Q), T = \{(s,a,q) \}$

$T(p)\in Q \to \mathbb{P}(V\times Q), T(p) = \{(a,q):(s,a,q) \in T\}$

\hfill

$p,q\in Q, a\in V$

$T:Q\times V\to Q$

$T(p,a)={q}$

\hfill

$Q_{map}:\mathbb{P}(Q\times V), Q_{map} = \{(q,a): (s,a,q) \in T\}$

$Q_{map}(q) = \{a: (s,a,q)\in T\}$

${Q_{map}}^{-1}: V \nrightarrow \mathbb{P}(Q), {Q_{map}}^{-1} = \{(a,q): (s,a,q) \in T\}$

\section{Properties of finite automata}

$M=(Q,V,T,E,S,F),M_0=(Q_0,V_0,T_0,E_0,S_0,F_0),M_1=(Q_1,V_1,T_1,E_1,S_1,F_1)$

\begin{definition}[Size of an $FA$] 
	Define the size of an $FA$ as $|M|=|Q|$
\end{definition}

\begin{definition}[Isomorphism同构 $(\cong)$ of $FA's$]
	We define isomorphism $(\cong)$ as an equivalence relation on FA's. $M_0$ and $M_1$ are isomorphic (written $M_0\cong M_1$) if and only if $V_0=V_1$ and there exists a bijection双射 $g\in Q_0\to Q_1$ such that
	\begin{itemize}
		\item $T_1=\{(g(p),a,g(q)|(p,a,q)\in T_0\}$
		\item $E_1=\{(g(p),g(q)|(p,q)\in E_0\}$
		\item $S_1=\{g(s)|s\in S_0\}$ and
		\item $F_1=\{g(f)|f\in F_0\}$
	\end{itemize} (see Fig \ref{fig:Isomorphism}). $\hfill\square$
\end{definition}

\begin{figure}[htbp]
	\centering
	\begin{tikzpicture}[>=latex, shorten >=1pt,node distance=0.75in, on grid, auto]
	\node[state,initial] (A0) {$s$};
	\node[state](A1) [right=of A0] {$p$};
	\node[state](A2) [right=of A1] {$q$};
	\node[state,accepting] (A3) [right=of A2] {$f$};
	\node[state,initial] (B0) [below=of A0] {$g(s)$};
	\node[state](B1) [right=of B0] {$g(p)$};
	\node[state](B2) [right=of B1] {$g(q)$};
	\node[state,accepting] (B3) [right=of B2] {$g(f)$};
	\path[->]
	(A0) edge node {$\cdots$} (A1)
	(A1) edge node {$a$} (A2)
	(A2) edge node {$\cdots$} (A3)
	(B0) edge node {$\cdots$} (B1)
	(B1) edge node {$a$} (B2)
	(B2) edge node {$\cdots$} (B3)
	(A0) edge node {$g$} (B0)
	(A1) edge node {$g$} (B1)
	(A2) edge node {$g$} (B2)
	(A3) edge node {$g$} (B3)
	;
	\end{tikzpicture}
	\caption{Isomorphism$M_0\cong M_1$ if and only if $V_0=V_1$ and there exists a bijection $g\in Q_0\to Q_1$}
	\label{fig:Isomorphism}	
\end{figure}

\begin{definition}[Extending the transition relation $T$]
	We extend transition relation $T\in V\to\mathbb{P}(Q\times Q)$ to $T^\ast\in V^\ast\to \mathbb{P}(Q\times Q)$ as follows:
	
	$T^\ast(\epsilon)=E^\ast$\\
	and (for $a\in V,w\in V^\ast$)
	
	$T^\ast(aw)=E^\ast\circ T(a)\circ T^\ast(w)$\\
	Operator $\circ$ (composition is defined in Convention \ref{con:comp}). 
	
	This definition could also have been presented symmetrically. $\hfill\square$	
\end{definition}

\begin{note} 
	$s_1,s_2,s_3,s_4\in Q,a\in V, w\in V^\ast$
	
	$E = T(\epsilon) = \{(s_1,s_2)\},T(a)=\{(s_2,s_3)\},T^\ast(w)=\{(s_3,s_4)\}$
	\begin{align*}
	   T^\ast(aw)&= E^\ast\circ T(a)\circ T^\ast(w) \\
	   &= \{(s_1,s_2)\} \circ \{(s_2,s_3)\} \circ \{(s_3,s_4)\}
	   &= \{(s_1,s_3)\} \circ \{(s_3,s_4)\}
	   &= \{(s_1,s_4)\}
	\end{align*}
\end{note}

\begin{note}
	$T\in Q\times V\to \mathbb{P}(Q)$,	extend to: $T^\ast \in Q\times V^\ast \to \mathbb{P}(Q)$
	
	$\forall q\in Q, w\in V^{\ast}, a\in V$,
	\begin{enumerate}
		\item $T^\ast(q,\epsilon) = q$
		\item $T^\ast(q,wa) = T(T^\ast(q,w),a)$
	\end{enumerate}
	\begin{align*}
	T^\ast(q,a) &= T^\ast(q,\epsilon a) \\
	&= T(T^\ast(q,\epsilon),a) \\
	&= T(q,a)
	\end{align*}
	两值相同,不用区分这两个符号。 
\end{note}

\begin{convention}[Relation composition]\label{con:comp} Given sets $A,B,C$ (not necessarily different) and two relations, $E\subseteq A\times B$ and $F\subseteq B\times C$, we define relation composition (infix operator中缀操作符 $\circ$) as:

$E\circ F = \{(a,c)|(\exists b\in B),(a,b)\in E\land (b,c)\in F)\}$  $\hfill\square$
\end{convention}

\begin{note} if $\exists b\in B,(a,b)\in E,(b,c)\in F,$ then
	
	$E:A\to B\Rightarrow E(a)=b$
	
	$F:B\to C\Rightarrow F(b)=c$
	
	$E\circ F =\{(a,b)\}\circ \{(b,c)\}=\{a,c\}$
	\begin{align*}
	(E\circ F)(a) &=F(E(a)) \\
	&=F(b) = c
	\end{align*}
\end{note}

\begin{remark}
	We also sometimes use the signature $T^\ast\in Q\times Q\to \mathbb{P}(V^\ast)$ $\hfill\square$
\end{remark}

\begin{note}
	$T(p,q)=\{w|p,q\in Q,w\in V^\ast\}$
\end{note}

\begin{remark}
	if $E=\emptyset$ then $E^\ast=\emptyset^\ast=I_Q$ where $I_Q$ is the identity relation单位关系 on the states of $M$.
\end{remark}

\begin{definition}[The language between states] 
	The language between any two states $q_0,q_1\in Q$ is $T^\ast(q_0,q_1)$. $\hfill\square$
\end{definition}

\begin{definition}[Left and right languages]
	The left language of a state (in $M$) is given by function, $\overleftarrow{L}_M\in Q\to\mathbb{P}(V^\ast)$,where
	
	$\overleftarrow{L}_M(q)=(\cup s:s\in S:T^\ast(s,q))$
	
	The right language of a state (in $M$) is given by function $\overrightarrow{L}_M\in Q\to\mathbb{P}(V^\ast)$, where
	
	$\overrightarrow{L}_M(q)=(\cup f:f\in F:T^\ast(q,f))$
	
	The subscript $M$ is usually dropped when no ambiguity can arise.$\hfill\square$
\end{definition}

\begin{example}
	$T^\ast\in Q\times Q\to \mathbb{P}(V^\ast)$,$\overleftarrow{L}_M,\overrightarrow{L}_M\in Q\to\mathbb{P}(V^\ast)$. 
	
	$\overleftarrow{L}_M(q)=$\{所有开始状态到$q$状态的字符串集合\}, (从$q$往左看)
	
	$\overrightarrow{L}_M(q)=$\{所有从$q$状态到接受状态的字符串集合\}, (从$q$往右看)
	
	\begin{tikzpicture}[>=latex, shorten >=1pt,node distance=0.75in, on grid, auto]
	\node[state,initial] (s) {$s$};
	\node[state](q) [right=of s] {$q$};
	\node[state,accepting] (f) [right=of q] {$f$};
	\path[->]
	(s) edge node {$\overleftarrow{L}_M(q)$} (q)
	(q) edge node {$\overrightarrow{L}_M(q)$} (f)
	;
	\end{tikzpicture} 
	
	see Fig \ref{fig:mod_53}.
	\begin{align*}
	   \overleftarrow{L}_M(q_2) &=(s\to q_1\to q_2) \cup (s\to (q_1\to q_3)^\ast\to q_1 \to q_2) \cup (s\to (q_1\to q_3)^\ast\to q_3 \to q_2) \\
	   &=[(s\to q_1\to q_2) \cup (s\to (q_1\to q_3)^\ast\to q_1 \to q_2)] \cup (s\to (q_1\to q_3)^\ast\to q_3 \to q_2) \\
	   & =(s\to (q_1\to q_3)^\ast\to q_1 \to q_2) \cup (s\to (q_1\to q_3)^\ast\to q_3 \to q_2) \\
	   &=\{1(10)^\ast 0,1(10)^\ast 1\}\\ 
	   \overrightarrow{L}_M(q_2) &=\{01^\ast0,10^\ast1(001^\ast 0 + (10)^\ast 1)\} 
	\end{align*}
	\begin{figure}[htbp] 
		%\sidecaption
		%\flushleft{$\overleftarrow{L}_M(q_2) =\{10,11,1(10)^\ast1\}$}
		\begin{tikzpicture}[>=latex, shorten >=1pt,node distance=0.75in, on grid, auto]
		\node[state] (q0) {$q_0$};
		\node[state,](q1) [below=of q0] {$q_1$};
		\node[state,accepting] (q3) [below left=of q1] {$q_3$};
		\node[state] (q2) [below right=of q1] {$q_2$};
		\node[state] (q4) [below left=of q2] {$q_4$};
		\node[state,initial] (s) [left=of q1] {$s$};
		\node[state](dump) [above=of s] {$d$};
		\path[->]
		(s) edge node {$0$} (dump)
		(s) edge node [bend left] {$1$} (q1)
		(q0) edge [loop above] node {$0$} (q0)
		(q0) edge [swap] node {$1$} (q1)
		(q1) edge node {$0$} (q2)
		(q2) edge node {$0$} (q4)
		(q4) edge node {$0$} (q3)
		(q3) edge [bend right,swap] node {$0$} (q1)
		(q1) edge [swap] node {$1$} (q3)
		(q2) edge [bend right] node {$1$} (q0)
		(q3) edge [bend right] node {$1$} (q2)
		(q4) edge [loop below] node {$1$} (q4)
		(dump) edge [loop above] node {$0,1$} (dump);
		\end{tikzpicture} 
		\caption{$\{x|x\in \{0,1\}^+ \text{且当把$x$看成二进制数时,$x$模5与3同余,}$\newline
			要求当$x$为0时,$|x|=1$,且当$x\ne 0$时,$x$的首字符为1\}语言对应的$DFA$}
		\label{fig:mod_53}
	\end{figure}
\end{example}

\begin{definition}[Language of an $FA$]
	The language of a finite automaton (with alphabet $V$) is given by the function $L_{FA}\to \mathbb{P}(V^\ast)$ defined as:
	
	$L_{FA}(M)=(\cup s,f:s\in S\land f\in F:T^\ast(s,f))$ (所有从开始状态到接受状态的字符串集合)
	$\hfill\square$
\end{definition}

\begin{property}[Language of an $FA$]
	From the definition of left and right languages (of a state), we can also write:
	
	$L_{FA}(M) = (\cup f:f\in F:\overleftarrow{L}(f))$ (所有从s到f的字符串集合,从f向左看)\\
	and
	
	$L_{FA}(M) = (\cup s:s\in S:\overrightarrow{L}(s))$ (所有从s到f的字符串集合,从s向右看) 
	$\hfill\square$
\end{property}

\begin{definition}[$\epsilon$-$free$ 无$\epsilon$转移]
	Automaton $M$ is $\epsilon$-$free$ if and only if $E=\emptyset$.$\hfill\square$
\end{definition}

\begin{remark}
	Even if $M$ is $\epsilon$-$free$ it is still possible that $\epsilon\in L_{FA}(M): in this case S\cap F \ne \emptyset$. (开始状态也是接受状态) $\hfill\square$
\end{remark}


Form \cite[Convention A.4]{WATSON93a} (Tuple projection).

\begin{convention}[Tuple projection]\label{con:project}
	For an $n$-tuple $t=(x_1,x_2,\dots,x_n)$ we use the notation
	$\pi_i(t) (1\le i\le n)$ to denote tuple element $x_i$; we use the notation $\bar{\pi}_i(t) (1\le i\le n)$ to denote the $(n-1)$-tuple $(x_1,\dots,x_{i-1},x_{i+1},\dots,x_n)$. Both $\pi$and $\bar{\pi}$ extend naturally to sets of tuples. $\hfill\square$
\end{convention}

Form \cite[Definition A.20]{WATSON93a} (Tuple and relation reversal).

\begin{definition}[Tuple and relation reversal] \label{def:reversal}
	For an $n$-tuple $(x_1,x_2,\dots,x_n)$ define reversal as (postfix and superscript) function R:
	
	$(x_1,x_2,\dots,x_n)^R = (x_n,x_n-1,\dots,x_2,x_1)$ \\ 
	Given a set $A$ of tuples, we define $A^R=\{x^R:x\in A\}. \hfill\square$
\end{definition}

\begin{definition}[Reachable states]
	For $M$ we can define a reachability relation $Reach(M)\subseteq (Q\times Q)$ defined as
	
	$Reach(M) = (\bar{\pi}_2(T)\cup E)^\ast$        see\footnote{\{$(p_1,q_1),(p_2,q_2),\dots$\}}\\
	Functions $\pi$ and $\bar{\pi}$ are defined in Convention \ref{con:project}. Similarly the set of start-reachable states is defined to be:
	
	$SReachable(M)=Reach(M)(S)$ see\footnote{从start state可以到达的状态集合}\\
	and the set of final-reachable states is defined to be:
	
	$FReachable(M)=(Reach(M))^R(F)$ see\footnote{可以到达final state的状态集合}\\
	Reversal of a relation is defined in Definition \ref{def:reversal}. The set of useful states is:
	$Reachable(M)=SReachable(M)\cap FReachable(M)$ $\hfill\square$
\end{definition}

\begin{remark} For $FA$ $M=(Q,V,T,E,S,F)$,function $SReachable$ satisfies the following interesting property:
	
	$q\in SReachable(M)\equiv\overleftarrow{L}_M(q)\ne\emptyset$\\
	$FReachable$ satisfies a similar property:
	
	$q\in FReachable(M)\equiv\overrightarrow{L}_M(q)\ne\emptyset\hfill\square$
\end{remark}

\begin{example} 
	$T\in \mathbb{P}(Q\times V\times Q), T = \{(p,a,q)|p,q\in Q,a\in V\}$,
	
	$\bar{\pi}_2(T) = \{(p,q)|(p,a,q)\in T \}$	
	
	$Q_{map} = (\bar{\pi}_1(T))^R, Q_{map} = \{(a,q)|(p,a,q) \in T\}^R = \{(q,a)|(p,a,q) \in T\}$
	
	
	$\hfill\square$
\end{example}


\begin{tikzpicture}[>=latex, shorten >=1pt,node distance=1.0in, on grid, auto]
\node[state,initial] (s) {$s$};
\node[state] (q0) [above right=of s] {$q_0$};
\node[state] (q1) [below right=of s] {$q_1$};
\node[state,accepting] (f) [below right =of q0]{$f$};
\path[->]
(s) edge node {$\overleftarrow{L}_M(q_0)$} (q0)
(s) edge [swap] node {$\overleftarrow{L}_M(q_1)$} (q1)
(q0) edge node {$\overrightarrow{L}_M(q_0)$} (f)
(q1) edge [swap] node {$\overrightarrow{L}_M(q_1)$} (f)
;
\end{tikzpicture} 

\hfill

e.g. $p=\{1,2\}\in Q_1\subseteq \mathbb{P}(Q_0),\overrightarrow{L}_{M_1}(p)=\overrightarrow{L}_{M_0}(1)\cup\overrightarrow{L}_{M_0}(2)$

\hfill

\begin{figure}[hbtp]
	\begin{tikzpicture}
		\matrix (a) [matrix of math nodes,row sep=3em,
		column sep=3em, nodes in empty cells]
		{ Q &  P(Q) \\ 
			&  D \\};
		\path[>=latex,->] 
		(a-1-1) edge node [auto] {$subset$} (a-1-2)
		        edge node [auto,swap] {$f$} (a-2-2)
		(a-1-2) edge node [auto] {$useful_s$} (a-2-2)
		;
	\end{tikzpicture}
	\caption{$subset\circ usefus_s= usefus_s(subset(Q,V,\emptyset,S,F))=(D,V,T^\prime,\emptyset,S^\prime,F^\prime)$}
\end{figure}

\section{$\Sigma$-algebras and regular expressions}

\noindent $X$集合中的元素与有序集$S$中的元素一一对应,称$X$是$S$-sorted.
$S=\{1,3,7,9\},X=\{d,a,c,f\},s\in S,X_s\in X$\\
$S$是有序的,$S_{s_1}=1,S_{s_2}=3,S_{s_3}=7,S_{s_4}=9$\\
$X$与$S$中的元素一一对应。$X_{s_1}=d,X_{s_2}=a,X_{s_3}=c,X_{s_4}=f$\\

\paragraph{$\Sigma$-homomorphism}
$\Sigma$-同态: $(V,F) \Leftrightarrow (W,G)$,载体$(V,W)$和操作$(F,G)$一一对应。

\noindent $V := \{re_1,re_2\}$ $RE$(正则表达式), $W := \{fa_1,fa_2\}$ $FA$(有限自动机)\\
$F:$RG运算,二元: union(or),concat; 一元: star,plus,question; 常量:epsilon,empty,symbol\\ 
$G:$FA运算, 同上\\
$V_{s_1}=re_1,V_{s_2} =re_2; W_{s_1}=fa_1,W_{s_2} =fa_2$

\begin{tikzpicture}[descr/.style={fill=white,inner sep=2.5pt}]
\matrix (m) [matrix of math nodes, row sep=3em,column sep=3em]
{ re_1 & re_2  & RE_{concat}(re_1,re_2)\\
  fa_1	& fa_2 & FA_{concat}(fa_1,fa_2)\\ };
\path[->,font=\scriptsize]
(m-1-1) edge node[descr] {$ h_{s_1} $} (m-2-1)
(m-1-2) edge node[descr] {$ h_{s_2} $} (m-2-2)
(m-1-3) edge node[auto,swap] {$ h_{s} $} (m-2-3)
;
\end{tikzpicture}


\begin{lstlisting}
//Sigma.h
template<class T>
class Reg :public T {
// Helper for constructing the homomorphic image of a regular expression.
// T is carrier set: RE,FA,RFA,
// 各自的操作,分别在Sig-RE.cpp,Sig-FA.cpp,Sig-RFA.cpp中定义
inline void homomorphic_image(const RE& r);
Reg<T>& epsilon();
Reg<T>& empty();
Reg<T>& symbol(const CharRange r);
Reg<T>& Or(const Reg<T>& r);
Reg<T>& concat(const Reg<T>& r);
Reg<T>& star();
Reg<T>& plus();
Reg<T>& question();
}
\end{lstlisting}

\section{Others}

\begin{definition}[Prefix-closure\cite{Chrison2007}]
	Let $L\subseteq V^*$,then
	$$\overline{L} := \{s\in V^*:(\exists t\in V^*)[st\in L]\}$$
\end{definition}
In words, the prefix closure of L is the language denoted by $\overline{L}$ and consisting of all the prefixes in L. In general, $L\subseteq \overline{L}$.

L is said to be prefix-closed if $L = \overline{L}$. Thus language L is prefix-closed if any prefix of any string in L is also an element of L.

$L_1 = \{\epsilon,a,aa\}, L_1 = \overline{L_1}, L_{1}$ is prefix-closed.

$L_2 = \{a,b,ab\}, \overline{L_2} = \{\epsilon,a,b,ab\}, L_2 \subset \overline{L_2}, L_2$ is not prefix closed.

\hfill

\begin{definition}[Post-closure\cite{Chrison2007}]
Let $L\subseteq V^{\ast}$ and $s\in L$. Then the post-language of L after s, denoted by L/s, is the language
$$ L/s := \{t\in V^{\ast}:st\in L\}$$
By definition, $L/s = \emptyset$ if $s \notin \overline{L}$.
\end{definition}

\hfill

\begin{definition}[Left derivatives\cite{WATSON93a}] Given language $A\subseteq V^{\ast}$ and $w\in V^{\ast}$ we define the left derivative of A with respect to $w$ as:
$$w^{-1}A = \{x\in V^{\ast}:wx\in A\}$$
$A$关于$w$的左导数,就是$A$中: \{$w$的后缀组成的字符串集合\}。

Sometimes derivatives are written as $D_{w}A$ or as $\frac{dA}{dw}$. Right derivatives are analogously defined. Derivatives can also be extended to $B^{-1}A$ where B is also a language.
\end{definition}

\begin{example}
$A = \{a,aab,baa\},a^{-1}A = D_{a}A = \frac{dA}{da} =\{\epsilon,ab,\emptyset\} = \{\epsilon,ab\}$ \hfill$\square$ 
\end{example}

\begin{example}
	$L = \{ba,baa,baab,ca\}, w = \{ba\},$
	
	$w^{-1}L =\{\epsilon,a,ab,\emptyset\} = \{\epsilon,a,ab\}$
	
	${(wa)}^{-1}L = {(baa)}^{-1}L = \{\emptyset,\epsilon,b,\emptyset\} = \{\epsilon,b\}$
	
	$a^{-1}(w^{-1}L) = a^{-1}\{\epsilon,a,ab\} = \{\emptyset,\epsilon,b\} = \{\epsilon,b\}$
	
	$w\in L \equiv \epsilon \in w^{-1}L,and {(wa)}^{-1}L = a^{-1}(w^{-1}L)$ \hfill$\square$ 
\end{example}

\begin{example}
	$a^{-1}\{a\} = \{\epsilon\}; \quad a^{-1}\{b\} = \emptyset,\quad\Leftarrow if (a\ne b)$ \hfill$\square$ 
\end{example}

\begin{example}
	$L_0 = \{ab\},L_1 = \{ac\}, L_0L_1 = \{abac\}$
	
	$a^{-1}(L_0L_1) = \{bac\}$
	
	$a^{-1}(L_0L_1) = (a^{-1}L_0)L_1 \cup \emptyset \quad \Leftarrow(\epsilon \notin L_0)$
	
	$= \{b\}L_1 = \{bac\}$ \hfill$\square$ 
\end{example}

\begin{example}
	$L_0 = \{\epsilon,ab\},L_1 = \{ac\}, L_0L_1 = \{ac,abac\}$
	
	$a^{-1}(L_0L_1) = \{c,bac\}$
	
	$a^{-1}(L_0L_1) = (a^{-1}L_0)L_1 \cup a^{-1}L_1 \quad\Leftarrow(\epsilon \in L_0)$
	
	$= \{\emptyset,b\}L_1 \cup \{c\} = \{c,bac\}$ \hfill$\square$ 
\end{example}

\begin{proof}
	$a^{-1}(L_0L_1)$
	
	$1. if(\epsilon \in L_0) \Rightarrow a^{-1}(L_0L_1) = (a^{-1}L_0)L_1 \cup a^{-1}L_1 $ 
	
	$L_0 = (L_0 \backslash \{\epsilon\}) \cup \{\epsilon\}$
	
	$a^{-1}(L_0L_1) = a^{-1}(((L_0 \backslash \{\epsilon\}) \cup \{\epsilon\})L_1)$
	
	$=a^{-1}(L_0L_1\cup L_1)$
	
	$a^{-1}L_0 = a^{-1}((L_0 \backslash \{\epsilon\}) \cup \{\epsilon\})$
	
	$=a^{-1}(L_0\ \backslash \{\epsilon \}) \cup a^{-1}\{\epsilon \}$
	
	$=a^{-1}L_0 \cup \emptyset = a^{-1}L_0$
\end{proof}
	

\clearpage

From \cite[p99]{Hopcroft2008}

(1)如果$L$是一个语言,$a$是一个符号,则$L/a$(称作$L$和$a$的商)是所有满足如下条件的串$w$的集合:$wa$属于$L$。
例如,如果$L=\{a,aab,baa\}$,则$L/a = \{\epsilon,ba\}$, 证明:如果$L$是正则的,那么$L/a$也是。提示:从$L$的$DFA$出发,考虑接受状态的集合。

(2)如果$L$是一个语言,$a$是一个符号,则$a\backslash L$是所有满足如下条件的串$w$的集合: $aw$属于$L$。例如,如果$L=\{a,aab,baa\}$,则$a\backslash L=\{\epsilon,ab\}$,证明:如果$L$是正则的,那么$a\backslash L$也是。提示:记得正则语言在反转运算下是封闭的,又由(1)知,正则语言的商运算下是封闭的。

\begin{definition}[Kleene-closure\cite{Chrison2007}]
	Let $L\subseteq V^{\ast}$, then 
	$$L^{\ast} := \{\epsilon\}\cup L \cup LL\cup LLL\cup \cdots$$ 
\end{definition}

This is the same operation that we defined above for the set V, except that now it is applied to set L whose elements may be strings of length greater than one. An element of $L^*$ is formed by the concatenation of a finite (but possibly arbitrarily large) number of elements of L; this includes the concatenation of "zero" elements, that is the empty string $\epsilon$. Note that $\ast$ operation is idempotent: ${(L^*)}^* = L^*$.
\begin{align*}
L^{\ast} &= \{\epsilon\} +L^{+} \\
&=  \{\epsilon\}\cup (L\backslash \{\epsilon\})L^{\ast}\\ &=\{\epsilon\} + L + LL + LLL + \cdots 
\end{align*}

\section{Linear equation}

see \cite[5.3,p64]{Jean2018}.

We give an algorithm to covert an automaton to a rational(regular) expression. The algorithm amounts to solving a system of linear equations on languages. We first consider an equation of the form
\begin{equation}\label{Linear equation}
X = KX + L
\end{equation}

\begin{proposition}[Arden's Lemma]
	if $K$ does not contain the empty word, then $X = K^\star L$ is the unique solution of the equation $X = KX + L$.
\end{proposition}
where $K$ and $L$ are languages and $X$ is the unknown. When $K$ does not contain the empty word, the equation admits a unique solution.

\begin{proof}
	Replacing $X$ by $K^{\ast}L$ in the expression $KX+L$,one gets
	$$K(K^{\ast})L + L = K^{+}L + L = (K^{+}L + L) = K^{\ast}L,$$
	and hence $X = K^{\ast}L$ is a solution of (\ref{Linear equation}). see\footnote[1]{\begin{align*}
		K^{\ast} &= \{\epsilon\} + K^{+} \\
		&=  \{\epsilon\} + (K\backslash \{\epsilon\})K^{\ast}\\ &=\{\epsilon\} + K + KK + KKK + \cdots 
		\end{align*}}\label{star}
	
	To Prove uniqueness, consider two solutions $X_{1}$ and $X_{2}$ of (\ref{Linear equation}). By symmetry,it suffices to show that each word $u$ of $X_1$ also belongs to $X_2$. Let us prove this result by induction on the length of $u$.
	
	If $|u| = 0$, $u$ is the empty word\footnote[2]{The empty word = $\epsilon,|\epsilon| = 0$; if a language $M = \{\epsilon\}, |M| = 1$, The empty language $M = \emptyset, |M| = 0$. 文献\cite{Jean2018}用1表示$\epsilon$,因为$\epsilon K = K\epsilon = K$, 因此,$\epsilon$是连接运算的单位元,正是1表示的用意。0表示$\emptyset$,它是并运算的单位元,$K\cup \emptyset = \emptyset \cup K = K$.} and if $u \in X_1 = KX_1 + L$, then necessarily $u\in L$ since $\epsilon \notin K$. But in this case, $u\in KX_2 + L = X_2$. see\footnote[3]{In this case,$|u| = 0, X = \{\epsilon\},|X| = 1.\quad i.e.\quad\epsilon = K\epsilon + L, \epsilon = K + L$}
	
	For the induction step, consider a word $u$ of $X_1$ of length $n + 1$. Since $X_1 = KX_1 + L, u$ belongs either to $L$ or to $KX_1$. if $u\in L$, then $u\in KX_2 + L = X_2$. If $u\in KX_1$ then $u = kx$ for some $k\in K$ and $x\in X_1$. Since $k$ is not the empty word, one has necessarily $|x| \leq n$ and hence by induction $x\in X_2$.[see\footnote[4]{$u = kx,|u|=|kx|=n+1,\epsilon \notin K,|k|\ge 1,|x|\le n$,由假设知,$u$属于$X_1$,归纳$|x|=0,|x|=1,\cdots,n,x\in X_2$.}] It follows that $u\in KX_2$ and finally $u\in X_2$. This conclude the induction and the proof of the proposition.  \hfill$\square$ 
\end{proof}

\begin{svgraybox}
	
From \cite[p74]{Wonham2018} 
The \textit{length} $|s|$ of a string $s\in \Sigma^{\ast}$ is defined according to
$$|\epsilon| = 0; |s| = k, \text{if }s = \sigma_1\sigma_2 \cdots\sigma_k\in \Sigma ^{+}$$
Thus $|cat(s,t)| = |s| + |t|$.

A \textit{language} over $\Sigma$ is any subset of $\Sigma^{\ast}$, i.e. an element of the power set $Pwr(\Sigma^{\ast})$; thus the definition includes both the empty language $\emptyset$, and $\Sigma^{\ast}$ itself.

Note the distinction between $\emptyset$ (the language with no strings) and $\epsilon$ (the string with no symbols). For instance the language $\{\epsilon\}$ is nonempty, but contains only the empty string.
\end{svgraybox}	

From \cite[p78]{Wonham2018}
\begin{proposition}[\cite{Wonham2018}]
	\begin{enumerate}
		\item If $L = M^{\ast}N$ then $L = ML + N$
		\item If $\epsilon\notin M$ then $L = ML + N$ implies $L = M^{\ast}N$ \hfill$\square$ 
	\end{enumerate}
\end{proposition}
Part(2) is Known as Arden's rule. Taken with Part(1) it says that if $\epsilon\notin M$ then $L = M^{\ast}N$ is the unique solution of $L = ML + N$; in particular if $L = ML$ (with $\epsilon \notin M$) then $L = \emptyset$

\begin{exercise}
	Show by counterexample that the restriction $\epsilon \notin M$ in Arden's rule cannot be dropped.
\end{exercise}
\begin{solution}
	Examples text goes here.  
\end{solution}

\begin{exercise}
	Prove Arden's rule. Hint: If $L = ML + N$ then for every $k\geq 0$
	$$L = M^{k+1}L + (M^{k} + M^{k-1} + \cdots + M + \epsilon)N$$
\end{exercise}

\begin{solution}
	\begin{align*} Preliminaries:\\
		M^{\ast} &= M^{k} + M^{k-1} + \cdots + M^{1} + M^{0} \qquad (k\geq 0)\\
		         &= M^{k} + M^{k-1} + \cdots + M^{1} + \epsilon \\ 
		         &= M^{+} + \epsilon\\
		         &= MM^{*} + \epsilon\\
		         &= (M\backslash\{\epsilon\})M^{*} + \epsilon\\
		M^{+} &= M^{k} + M^{k-1} + \cdots + M^{1} \qquad (k > 0)\\
		      &= M(M^{k} + M^{k-1} + \cdots + M^{1} + M^{0}) \\ 
		      &= MM^{*}\\
	    M^{0} &= \{\epsilon\} = 1\\
	    M\epsilon &= \epsilon M = M\\
	    \epsilon+\epsilon &=\epsilon\\
	    M+M &= M
	\end{align*}
	\begin{proof}
		$$L = ML + N \Rightarrow$$  
		\begin{align}
			M^0L &= M^1L + M^0N \\
			M^1L &= M^2L + M^1N \\ 
			M^2L &= M^3L + M^2N \\ 
		\end{align}
		$$\cdots$$
		$\Rightarrow\\ (M^0+M^1+M^2+\cdots)L = (M^1+M^2+M^3+\cdots)L + (M^0+M^1+M^2+\cdots)N \\ \Rightarrow$ 
	    
	    so,if $L = ML +N$,then for every $k \ge 0$
	    
		$L = M^{k+1}L + (M^k+M^{k-1}+\cdots+M+M^0)N \\ \Rightarrow$
		
		\begin{equation}\label{leq}
			L = M^{k+1}L + (M^k+M^{k-1}+\cdots+M+\epsilon)N
		\end{equation}
		
		(1) $k = 0$\\
		$L = ML + (\epsilon)N = ML + N \\ \Rightarrow 
		(1-M)L = N \\
		(\epsilon-M)L = N$\\
		由于$\epsilon \notin M$,左端不会消去\{$\epsilon$\}.因此,只能在$N$中找$L$,仅有唯一解: $L
		=\{\epsilon\} = \text{\{empty word\}} \subseteq N$.  

	\end{proof}
%	Replacing L by $M^{\ast}N$ in the expression $ML + N$, one gets
%	\begin{align*}
%	     $L = ML + N$
%	\end{align*}
%	and hence $L = M^{*}N$ is a solution of $L = ML + N$. 
%	
%	To Prove uniqueness, consider two solutions $L_{1}$ and $L_{2}$ of $ML + N$. By symmetry,it suffices to show that each string $s$ of $L_1$ also belongs to $L_2$. Let us prove this result by induction on the length of $s$.
%	
%	if $|s| = 0$, $s$ is the empty string($\epsilon$) and if $s\in L_1 = ML_1 + N$, then necessarily $s\in N$ since $\epsilon\notin M$. But in this case, $s\in ML_2 + N = L_2$. see\footnote[1]{In this case,$|s| = 0, L = \{\epsilon\},|L| = 1.\quad i.e.\quad\epsilon = M\epsilon + N,  \epsilon = M + N$}
%	
%	Replacing L by $M^{\ast}N$ in the expression $ML + N$, one gets
%	\begin{align*}
%	L = M(M^{\ast}N) + N &= M((M^{k} + M^{k-1} + \cdots + M^{1} + M^{0})N) + N \qquad (k\geq 0)\\
%	&= \text{(by associative law)} \\ 
%	&= (M(M^{k} + M^{k-1} + \cdots + M^{1} + M^{0}))N + N \\
%	&= \text{(by distributive law)} \\ 
%	&= (M^{k+1} + M^{k} + M^{k-1} + \cdots + M^{1})N + N \\
%	&= \text{(by distributive law)} \\ 
%	&= M^{k+1}N + (M^{k} + M^{k-1} + \cdots + M^{1})N + N \\
%	&= M^{k+1}N + (M^{k} + M^{k-1} + \cdots + M^{1} + 1)N \\
%	&= \text{by }(M^{0} = \epsilon = 1) \\
%	&= M^{k+1}N + (M^{k} + M^{k-1} + \cdots + M^{1} + \epsilon)N \\
%	&= M^{k+1}N + M^{*}N \\
%	&= \text{(by $L = M^{*}N$)} \\
%	&= M^{k+1}N + L
%	\end{align*}  
%	$k\geq 0$
%	$$L = M^{k+1}L + (M^{k} + M^{k-1} + \cdots + M + \epsilon)N$$ 

\end{solution}

From \cite[definition 2.3]{R.Su and Wonham2004}

\begin{definition}\label{def:epimorphic} Let
	
	$G_A=(X_A,\Sigma,\xi_A,x_{A,0},X_{A,m})$
	
	$G_A=(X_B,\Sigma,\xi_B,x_{B,0},X_{B,m})$\\
	$G_B$ is a DES-epimorphic image(满射像) of $G_A$ under DES-epimorphism $\theta: X_A \to X_B$ if
	\begin{enumerate}
		\item $\theta:X_A\to X_B$ is surjective(满射)
		\item $\theta(x_{A,0})=x_{B,0}$ and $\theta(X_{A,m})=X_{B,m}$
		\item $(\forall x\in X_A)(\forall \sigma\in\Sigma)\xi_A(x,\sigma)! \Rightarrow [\xi_B(\theta(x),\sigma)!\&\xi_B(\theta(x),\sigma)=\theta(\xi_A(x,\theta))]$ \label{item3}
		\item $(\forall x\in X_B)(\forall\sigma\in\Sigma)\xi_B(x,\sigma)!\Rightarrow [(\exists x^\prime\in X_A)\xi_A(x^\prime,\sigma)!\&\theta(x^\prime)=x]$ \label{item4}
	\end{enumerate}
	
	In particular,$G_B$ is DES-isomorphic(同构) to $G_A$ if $\theta:X_A\to X_B$ is bijective(双射).
	
	see figure \ref{fig:epimorphic}.
\end{definition}

\begin{figure}[htbp]
	\centering
	\flushleft{$\theta(x_{A,0})=x_{B,0}$ and $\theta(X_{A,m})=X_{B,m}$\\
	$\theta(x_A)=x_B$ and $\theta(x_A^\prime)=x_B^\prime$\\
	$\xi_A(x_A,\sigma)=x_A^\prime$ and $\xi_B(x_B,\sigma)=x_B^\prime \Rightarrow$  definition \ref{def:epimorphic} (\ref{item3},\ref{item4}) 
	\\
    } 
	\begin{tikzpicture}[>=latex, shorten >=1pt,node distance=0.75in, on grid, auto]
	\node[state,initial] (A0) {$x_{A,0}$};
	\node[state](A1) [right=of A0] {$x_A$};
	\node[state](A2) [right=of A1] {$x_A^\prime$};
	\node[state,accepting] (A3) [right=of A2] {$X_{A,m}$};
	\node[state,initial] (B0) [below=of A0]{$x_{B,0}$};
	\node[state](B1) [right=of B0] {$x_B$};
	\node[state](B2) [right=of B1] {$x_B^\prime$};
	\node[state,accepting] (B3) [right=of B2] {$X_{B,m}$};
	\path[->]
	(A0) edge node {$\cdots$} (A1)
	(A1) edge node {$\sigma$} (A2)
	(A2) edge node {$\cdots$} (A3)
	(B0) edge node {$\cdots$} (B1)
	(B1) edge node {$\sigma$} (B2)
	(B2) edge node {$\cdots$} (B3)
	(A0) edge node {$\theta$} (B0)
	(A1) edge node {$\theta$} (B1)
	(A2) edge node {$\theta$} (B2)
	(A3) edge node {$\theta$} (B3)
	;
	\end{tikzpicture}
	\caption{definition \ref{def:epimorphic}, $G_B$ is a DES-epimorphic image(满射像) of $G_A$ under DES-epimorphism $\theta: X_A \to X_B$ }
	\label{fig:epimorphic}	
\end{figure}

%%%%%%%%%%%%%%%%%%%%%%%%% referenc.tex %%%%%%%%%%%%%%%%%%%%%%%%%%%%%%
% sample references
% %
% Use this file as a template for your own input.
%
%%%%%%%%%%%%%%%%%%%%%%%% Springer-Verlag %%%%%%%%%%%%%%%%%%%%%%%%%%
%
% BibTeX users please use
% \bibliographystyle{}
% \bibliography{}
%
\biblstarthook{In view of the parallel print and (chapter-wise) online publication of your book at \url{www.springerlink.com} it has been decided that -- as a genreral rule --  references should be sorted chapter-wise and placed at the end of the individual chapters. However, upon agreement with your contact at Springer you may list your references in a single seperate chapter at the end of your book. Deactivate the class option \texttt{sectrefs} and the \texttt{thebibliography} environment will be put out as a chapter of its own.\\\indent
References may be \textit{cited} in the text either by number (preferred) or by author/year.\footnote{Make sure that all references from the list are cited in the text. Those not cited should be moved to a separate \textit{Further Reading} section or chapter.} The reference list should ideally be \textit{sorted} in alphabetical order -- even if reference numbers are used for the their citation in the text. If there are several works by the same author, the following order should be used: 
\begin{enumerate}
\item all works by the author alone, ordered chronologically by year of publication
\item all works by the author with a coauthor, ordered alphabetically by coauthor
\item all works by the author with several coauthors, ordered chronologically by year of publication.
\end{enumerate}
The \textit{styling} of references\footnote{Always use the standard abbreviation of a journal's name according to the ISSN \textit{List of Title Word Abbreviations}, see \url{http://www.issn.org/en/node/344}} depends on the subject of your book:
\begin{itemize}
\item The \textit{two} recommended styles for references in books on \textit{mathematical, physical, statistical and computer sciences} are depicted in ~\cite{science-contrib, science-online, science-mono, science-journal, science-DOI} and ~\cite{phys-online, phys-mono, phys-journal, phys-DOI, phys-contrib}.
\item Examples of the most commonly used reference style in books on \textit{Psychology, Social Sciences} are~\cite{psysoc-mono, psysoc-online,psysoc-journal, psysoc-contrib, psysoc-DOI}.
\item Examples for references in books on \textit{Humanities, Linguistics, Philosophy} are~\cite{humlinphil-journal, humlinphil-contrib, humlinphil-mono, humlinphil-online, humlinphil-DOI}.
\item Examples of the basic Springer style used in publications on a wide range of subjects such as \textit{Computer Science, Economics, Engineering, Geosciences, Life Sciences, Medicine, Biomedicine} are ~\cite{basic-contrib, basic-online, basic-journal, basic-DOI, basic-mono}. 
\end{itemize}
}

\begin{thebibliography}{99.}%
% and use \bibitem to create references.
%
% Use the following syntax and markup for your references if 
% the subject of your book is from the field 
% "Mathematics, Physics, Statistics, Computer Science"
%
% Contribution 
\bibitem{science-contrib} Broy, M.: Software engineering --- from auxiliary to key technologies. In: Broy, M., Dener, E. (eds.) Software Pioneers, pp. 10-13. Springer, Heidelberg (2002)
%
% Online Document
\bibitem{science-online} Dod, J.: Effective substances. In: The Dictionary of Substances and Their Effects. Royal Society of Chemistry (1999) Available via DIALOG. \\
\url{http://www.rsc.org/dose/title of subordinate document. Cited 15 Jan 1999}
%
% Monograph
\bibitem{science-mono} Geddes, K.O., Czapor, S.R., Labahn, G.: Algorithms for Computer Algebra. Kluwer, Boston (1992) 
%
% Journal article
\bibitem{science-journal} Hamburger, C.: Quasimonotonicity, regularity and duality for nonlinear systems of partial differential equations. Ann. Mat. Pura. Appl. \textbf{169}, 321--354 (1995)
%
% Journal article by DOI
\bibitem{science-DOI} Slifka, M.K., Whitton, J.L.: Clinical implications of dysregulated cytokine production. J. Mol. Med. (2000) doi: 10.1007/s001090000086 
%
\bigskip

% Use the following (APS) syntax and markup for your references if 
% the subject of your book is from the field 
% "Mathematics, Physics, Statistics, Computer Science"
%
% Online Document
\bibitem{phys-online} J. Dod, in \textit{The Dictionary of Substances and Their Effects}, Royal Society of Chemistry. (Available via DIALOG, 1999), 
\url{http://www.rsc.org/dose/title of subordinate document. Cited 15 Jan 1999}
%
% Monograph
\bibitem{phys-mono} H. Ibach, H. L\"uth, \textit{Solid-State Physics}, 2nd edn. (Springer, New York, 1996), pp. 45-56 
%
% Journal article
\bibitem{phys-journal} S. Preuss, A. Demchuk Jr., M. Stuke, Appl. Phys. A \textbf{61}
%
% Journal article by DOI
\bibitem{phys-DOI} M.K. Slifka, J.L. Whitton, J. Mol. Med., doi: 10.1007/s001090000086
%
% Contribution 
\bibitem{phys-contrib} S.E. Smith, in \textit{Neuromuscular Junction}, ed. by E. Zaimis. Handbook of Experimental Pharmacology, vol 42 (Springer, Heidelberg, 1976), p. 593
%
\bigskip
%
% Use the following syntax and markup for your references if 
% the subject of your book is from the field 
% "Psychology, Social Sciences"
%
%
% Monograph
\bibitem{psysoc-mono} Calfee, R.~C., \& Valencia, R.~R. (1991). \textit{APA guide to preparing manuscripts for journal publication.} Washington, DC: American Psychological Association.
%
% Online Document
\bibitem{psysoc-online} Dod, J. (1999). Effective substances. In: The dictionary of substances and their effects. Royal Society of Chemistry. Available via DIALOG. \\
\url{http://www.rsc.org/dose/Effective substances.} Cited 15 Jan 1999.
%
% Journal article
\bibitem{psysoc-journal} Harris, M., Karper, E., Stacks, G., Hoffman, D., DeNiro, R., Cruz, P., et al. (2001). Writing labs and the Hollywood connection. \textit{J Film} Writing, 44(3), 213--245.
%
% Contribution 
\bibitem{psysoc-contrib} O'Neil, J.~M., \& Egan, J. (1992). Men's and women's gender role journeys: Metaphor for healing, transition, and transformation. In B.~R. Wainrig (Ed.), \textit{Gender issues across the life cycle} (pp. 107--123). New York: Springer.
%
% Journal article by DOI
\bibitem{psysoc-DOI}Kreger, M., Brindis, C.D., Manuel, D.M., Sassoubre, L. (2007). Lessons learned in systems change initiatives: benchmarks and indicators. \textit{American Journal of Community Psychology}, doi: 10.1007/s10464-007-9108-14.
%
%
% Use the following syntax and markup for your references if 
% the subject of your book is from the field 
% "Humanities, Linguistics, Philosophy"
%
\bigskip
%
% Journal article
\bibitem{humlinphil-journal} Alber John, Daniel C. O'Connell, and Sabine Kowal. 2002. Personal perspective in TV interviews. \textit{Pragmatics} 12:257--271
%
% Contribution 
\bibitem{humlinphil-contrib} Cameron, Deborah. 1997. Theoretical debates in feminist linguistics: Questions of sex and gender. In \textit{Gender and discourse}, ed. Ruth Wodak, 99--119. London: Sage Publications.
%
% Monograph
\bibitem{humlinphil-mono} Cameron, Deborah. 1985. \textit{Feminism and linguistic theory.} New York: St. Martin's Press.
%
% Online Document
\bibitem{humlinphil-online} Dod, Jake. 1999. Effective substances. In: The dictionary of substances and their effects. Royal Society of Chemistry. Available via DIALOG. \\
http://www.rsc.org/dose/title of subordinate document. Cited 15 Jan 1999
%
% Journal article by DOI
\bibitem{humlinphil-DOI} Suleiman, Camelia, Daniel C. O�Connell, and Sabine Kowal. 2002. `If you and I, if we, in this later day, lose that sacred fire...�': Perspective in political interviews. \textit{Journal of Psycholinguistic Research}. doi: 10.1023/A:1015592129296.
%
%
%
\bigskip
%
%
% Use the following syntax and markup for your references if 
% the subject of your book is from the field 
% "Computer Science, Economics, Engineering, Geosciences, Life Sciences"
%
%
% Contribution 
\bibitem{basic-contrib} Brown B, Aaron M (2001) The politics of nature. In: Smith J (ed) The rise of modern genomics, 3rd edn. Wiley, New York 
%
% Online Document
\bibitem{basic-online} Dod J (1999) Effective Substances. In: The dictionary of substances and their effects. Royal Society of Chemistry. Available via DIALOG. \\
\url{http://www.rsc.org/dose/title of subordinate document. Cited 15 Jan 1999}
%
% Journal article by DOI
\bibitem{basic-DOI} Slifka MK, Whitton JL (2000) Clinical implications of dysregulated cytokine production. J Mol Med, doi: 10.1007/s001090000086
%
% Journal article
\bibitem{basic-journal} Smith J, Jones M Jr, Houghton L et al (1999) Future of health insurance. N Engl J Med 965:325--329
%
% Monograph
\bibitem{basic-mono} South J, Blass B (2001) The future of modern genomics. Blackwell, London 
%
\end{thebibliography}


\begin{thebibliography}{99}
	\bibitem[Hopcroft2008]{Hopcroft2008}
	John E. Hopcroft,Rajeev Motwani,Jeffrey D. Ullman著,孙家骕等译,\textit{自动机理论、语言和计算机导论},Third Edition, 机械工业出版社,2008.7
	
	\bibitem[WATSON93a]{WATSON93a}
	WATSON, B. W. \textit{A taxonomy of finite automata construction algorithms}, Computing Science Note 93/43, Eindhoven University of Technology, The Netherlands, 1993. Available by ftp from ftp.win.tue.nl in pub/techreports/pi.
	
	\bibitem[WATSON93b]{WATSON93b}
	WATSON, B. W. \textit{A taxonomy of finite automata minimization algorithms}, Computing Science Note 93/44, Eindhoven University of Technology, The Netherlands, 1993. Available by ftp from ftp.win.tue.nl in pub/techreports/pi.
	
	\bibitem[WATSON94a]{WATSON94a}
	WATSON, B. W. \textit{An introduction to the FIRE engine: A C++ toolkit for FInite automata and Regular Expressions}, Computing Science Note 94/21, Eindhoven University of Technology, The Netherlands, 1994. Available by ftp from ftp.win.tue.nl in pub/techreports/pi
	
	\bibitem[WATSON94b]{WATSON94b}
	WATSON, B.W. \textit{The design. and implementation of the FIRE engine:	A C++ toolkit for FInite automata and Regular Expressions}, Computing Science Note 94/22, Eindhoven University of Technology, The Netherlands, 1994. Available by ftp from ftp.win.tue.nl in pub/techreports/pi.
	
	\bibitem[Chrison2007]{Chrison2007}
	Christos G. Cassandras and St$\acute{e}$phane Lafortune, \textit{Introduction to Discrete Event Systems},Second Edition,New York,Springer,2007
	
	\bibitem[Wonham2018]{Wonham2018}
	W. M. Wonham and Kai Cai,\textit{Supervisory Control of Discrete-Event Systems}, Revised 2018.01.01
	
	\bibitem[Jean2018]{Jean2018}
	Jean-$\acute{E}$ric Pin, \textit{Mathematical Foundations of Automata Theory},Version of June 15,2018
	
	\bibitem[蒋宗礼2013]{蒋宗礼2013}
	蒋宗礼,姜守旭, \textit{形式语言与自动机理论(第3版)}, 清华大学出版社,2013.05
	
	
	\bibitem[Lipschutz2007]{Lipschutz2007}
	S. Lipschutz and M. L. Lipson, \textit{Schaum's Outline of Theory and Problems of Discrete Mathematics}, Third Edition, New York: McGraw-Hill, 2007.
	
	\bibitem[Rosen2007]{Rosen2007}
	K. H. Rosen, \textit{Discrete Mathematics and Its Applications}, Seventh Edition, New York: McGraw-Hill, 2007.
	
	\bibitem[R.Su and Wonham2004]{R.Su and Wonham2004}
	R. Su and W. M. Wonham, \textit{Supervisor reduction for discrete-event systems}, Discrete Event Dyn. Syst., vol. 14, no. 1, pp. 31–53, Jan. 2004.
	
	\bibitem{Maclane1988}
	S. Maclane and G. Birkhoff, \textit{Algebra}, Third Edition, New York: Bchelsea Publishing Company, 1988.
\end{thebibliography}

