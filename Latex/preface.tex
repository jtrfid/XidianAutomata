%%%%%%%%%%%%%%%%%%%%%%preface.tex%%%%%%%%%%%%%%%%%%%%%%%%%%%%%%%%%%%%%%%%%
% sample preface
%
% Use this file as a template for your own input.
%
%%%%%%%%%%%%%%%%%%%%%%%% Springer %%%%%%%%%%%%%%%%%%%%%%%%%%

\preface

%% Please write your preface here
%Use the template \emph{preface.tex} together with the Springer document class SVMono (monograph-type books) or SVMult (edited books) to style your preface in the Springer layout.

%A preface\index{preface} is a book's preliminary statement, usually written by the \textit{author or editor} of a work, which states its origin, scope, purpose, plan, and intended audience, and which sometimes includes afterthoughts and acknowledgments of assistance.

%When written by a person other than the author, it is called a foreword. The preface or foreword is distinct from the introduction, which deals with the subject of the work.

%Customarily \textit{acknowledgments} are included as last part of the preface.

The origin of this note is from the course \textit{Introduction to Discrete Event Systems} with numbers X18ME1122 and Z04ME1216 for postgraduates of the School of Electro-Mechanical Engineering, Xidian University. Due to the diversity of content presented in this course, it is overwhelmingly mandatory to have a note that collects typical methodologies of discrete event systems, which matches the main research interests of \textit{Systems Control \& Automation} Group.


\vspace{\baselineskip}
\begin{flushright}\noindent
Xi'an, China\hfill\\ %{\it Firstname  Surname}\\
June 2017\hfill {\it Zhiwu Li}\\
\end{flushright}


