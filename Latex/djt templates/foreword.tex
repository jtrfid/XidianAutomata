%%%%%%%%%%%%%%%%%%%%%%foreword.tex%%%%%%%%%%%%%%%%%%%%%%%%%%%%%%%%%
% sample foreword
%
% Use this file as a template for your own input.
%
%%%%%%%%%%%%%%%%%%%%%%%% Springer %%%%%%%%%%%%%%%%%%%%%%%%%%

\foreword

%% Please have the foreword written here
%Use the template \textit{foreword.tex} together with the Springer document class SVMono (monograph-type books) or SVMult (edited books) to style your foreword\index{foreword} in the Springer layout.

%The foreword covers introductory remarks preceding the text of a book that are written by a \textit{person other than the author or editor} of the book. If applicable, the foreword precedes the preface which is written by the author or editor of the book.

Recent years have seen striking development of the notions of discrete event systems that permeate real-world technological complexes that are in general computer-integrated. Petri nets and automata are thought of as two major mathematical formalisms to address many problems arising from discrete event systems, pertaining to modeling, analysis, control, scheduling, and performance evaluation of discrete event systems. This note aims to provide a systematic yet fundamental treatment of the preliminaries of discrete event systems.

\vspace{\baselineskip}
\begin{flushright}\noindent
Xi'an, China, April, 2017\hfill {\it Zhiwu Li}\\
\end{flushright}


