%%%%%%%%%%%%%%%%%%%%% chapter.tex %%%%%%%%%%%%%%%%%%%%%%%%%%%%%%%%%
%
% sample chapter
%
% Use this file as a template for your own input.
%
%%%%%%%%%%%%%%%%%%%%%%%% Springer-Verlag %%%%%%%%%%%%%%%%%%%%%%%%%%
%\motto{Use the template \emph{chapter.tex} to style the various elements of your chapter content.}
\chapter{chapter format}
\label{intro} % Always give a unique label
% use \chaptermark{}
% to alter or adjust the chapter heading in the running head

\abstract{This paragraph shall summarize the contents of the paper in short terms.}

\paragraph{first Paragraph}
\subparagraph{first Subparagraph} 
\section{first section}
\label{first_section} % Always give a unique label
\subsection{first subsection}

\begin{svgraybox}
	emphasize complete paragraphs of texts 
\end{svgraybox}

description
\begin{description}[Type 1]
	\item[Type 1]{description type 1.}
	\item[Type 2]{description type 2.}
\end{description}

Problems or Exercises should be sorted chapterwise
\section*{Problems}
\addcontentsline{toc}{section}{Problems}
%
% Use the following environment.
% Don't forget to label each problem;
% the label is needed for the solutions' environment
\begin{prob}
	\label{prob1}
	A given problem or Excercise is described here. 
\end{prob}

\begin{prob}
	\label{prob2}
	\textbf{Problem Heading}\\
	(a) The first part of the problem is described here.\\
	(b) The second part of the problem is described here.
\end{prob}


\section{theorem/definition/proof/example}
\begin{theorem}
	Theorem text goes here.
\end{theorem}

%
\begin{definition}[DfName]
	Definition text goes here.
\end{definition}

\begin{proof}
	%\smartqed
	Proof text goes here.
	\qed
\end{proof}

\begin{example}
	Examples text goes here.  \hfill$\square$ 
\end{example}

\begin{corollary}
	Examples text goes here.  \hfill$\square$ 
\end{corollary}

\begin{lemma}
	Examples text goes here.  \hfill$\square$ 
\end{lemma}

\begin{remark}
	Examples text goes here.  \hfill$\square$ 
\end{remark}

\begin{proposition}
	Examples text goes here.  \hfill$\square$ 
\end{proposition}

\begin{note}
	Examples text goes here.
\end{note}

\begin{problem}
	Examples text goes here.
\end{problem}

\begin{question}
	Examples text goes here.
\end{question}

\begin{exercise}
	Examples text goes here.  
\end{exercise}

\begin{solution}
	Examples text goes here.  
\end{solution}


\section{equation}
equations, e.g.
\begin{equation}
a \times b = c\;,
\end{equation}

multiline equations
\begin{eqnarray}
a \times b = c \nonumber\\
\vec{a} \cdot \vec{b}=\vec{c}
\label{eq:01}
\end{eqnarray}

enumerate example
\begin{enumerate}
	\item{item 1.}
	\begin{enumerate}
		\item{sub item a.}
		\item{sub item b.}
	\end{enumerate}
	\item{item 2.}
\end{enumerate}

itemize example
\begin{itemize}
	\item{item 1, cf. Table~\ref{tab:1}.}
	\begin{itemize}
		\item{sub item 1.}
		\item{sub item 2.}
	\end{itemize}
	\item{item.}
\end{itemize}

For tables use
\begin{table}
	\caption{Please write your table caption here}
	\label{tab:1}       % Give a unique label
	%
	% For LaTeX tables use
	%
	\begin{tabular}{p{2cm}p{2.4cm}p{2cm}p{4.9cm}}
		\hline\noalign{\smallskip}
		Classes & Subclass & Length & Action Mechanism  \\
		\noalign{\smallskip}\svhline\noalign{\smallskip}
		Translation & mRNA$^a$  & 22 (19--25) & Translation repression, mRNA cleavage\\
		Translation & mRNA cleavage & 21 & mRNA cleavage\\
		Translation & mRNA  & 21--22 & mRNA cleavage\\
		Translation & mRNA  & 24--26 & Histone and DNA Modification\\
		\noalign{\smallskip}\hline\noalign{\smallskip}
	\end{tabular}
	$^a$ Table foot note (with superscript)
\end{table}

\section{figure}

For figures use
\begin{figure}[b]
\sidecaption
% Use the relevant command for your figure-insertion program
% to insert the figure file.
% For example, with the option graphics use
\includegraphics[scale=.65]{figure}
%
% If not, use
%\picplace{5cm}{2cm} % Give the correct figure height and width in cm
%
\caption{If the width of the figure is less than 7.8 cm use the \texttt{sidecapion} command to flush the caption on the left side of the page. If the figure is positioned at the top of the page, align the sidecaption with the top of the figure -- to achieve this you simply need to use the optional argument \texttt{[t]} with the \texttt{sidecaption} command}
\label{fig:1}       % Give a unique label
\end{figure}


\begin{figure}[t]
\sidecaption[t]
% Use the relevant command for your figure-insertion program
% to insert the figure file.
% For example, with the option graphics use
\includegraphics[scale=.65]{figure}
%
% If not, use
%\picplace{5cm}{2cm} % Give the correct figure height and width in cm
%
\caption{Please write your figure caption here}
\label{fig:2}       % Give a unique label
\end{figure}


\section{ref,cite}
\label{sec:2}
% Always give a unique label
% and use \ref{<label>} for cross-references
% and \cite{<label>} for bibliographic references
% use \sectionmark{}
% to alter or adjust the section heading in the running head

referenc Sect.~\ref{sec:2} , see also Fig.~\ref{fig:1}

referenc:Table. ~\ref{tab:1}

\cite{Lipschutz2007}

\cite[chap.2]{Lipschutz2007}


\runinhead{Run-in Heading Boldface Version} Use the \LaTeX\ automatism for all your cross-references and citations.
\subruninhead{Run-in Heading Italic Version} Use the \LaTeX\ automatism for all your cross-refer\-ences and citations as has already been described in Sect.~\ref{sec:2}\index{paragraph}.
% Use the \index{} command to code your index words
%

\begin{quotation}
	引用原文-- it will automatically render Springer's preferred layout.
\end{quotation}

footnote environment \footnote{脚注footnote.}.

参考文献
%%%%%%%%%%%%%%%%%%%%%%%%% referenc.tex %%%%%%%%%%%%%%%%%%%%%%%%%%%%%%
% sample references
% %
% Use this file as a template for your own input.
%
%%%%%%%%%%%%%%%%%%%%%%%% Springer-Verlag %%%%%%%%%%%%%%%%%%%%%%%%%%
%
% BibTeX users please use
% \bibliographystyle{}
% \bibliography{}
%
%\biblstarthook{In view of the parallel print and (chapter-wise) online publication of your book at \url{www.springerlink.com} it has been decided that -- as a genreral rule --  references should be sorted chapter-wise and placed at the end of the individual chapters. However, upon agreement with your contact at Springer you may list your references in a single seperate chapter at the end of your book. Deactivate the class option \texttt{sectrefs} and the \texttt{thebibliography} environment will be put out as a chapter of its own.\\\indent
%References may be \textit{cited} in the text either by number (preferred) or by author/year.\footnote{Make sure that all references from the list are cited in the text. Those not cited should be moved to a separate \textit{Further Reading} section or chapter.} The reference list should ideally be \textit{sorted} in alphabetical order -- even if reference numbers are used for the their citation in the text. If there are several works by the same author, the following order should be used: 
%\begin{enumerate}
%\item all works by the author alone, ordered chronologically by year of publication
%\item all works by the author with a coauthor, ordered alphabetically by coauthor
%\item all works by the author with several coauthors, ordered chronologically by year of publication.
%\end{enumerate}
%The \textit{styling} of references\footnote{Always use the standard abbreviation of a journal's name according to the ISSN \textit{List of Title Word Abbreviations}, see \url{http://www.issn.org/en/node/344}} depends on the subject of your book:
%\begin{itemize}
%\item The \textit{two} recommended styles for references in books on \textit{mathematical, physical, statistical and computer sciences} are depicted in ~\cite{science-contrib, science-online, science-mono, science-journal, science-DOI} and ~\cite{phys-online, phys-mono, phys-journal, phys-DOI, phys-contrib}.
%\item Examples of the most commonly used reference style in books on \textit{Psychology, Social Sciences} are~\cite{psysoc-mono, psysoc-online,psysoc-journal, psysoc-contrib, psysoc-DOI}.
%\item Examples for references in books on \textit{Humanities, Linguistics, Philosophy} are~\cite{humlinphil-journal, humlinphil-contrib, humlinphil-mono, humlinphil-online, humlinphil-DOI}.
%\item Examples of the basic Springer style used in publications on a wide range of subjects such as \textit{Computer Science, Economics, Engineering, Geosciences, Life Sciences, Medicine, Biomedicine} are ~\cite{basic-contrib, basic-online, basic-journal, basic-DOI, basic-mono}. 
%\end{itemize}
%}

\begin{thebibliography}{99}
	\bibitem[Hopcroft2008]{Hopcroft2008}
	John E. Hopcroft,Rajeev Motwani,Jeffrey D. Ullman?,?????,\textit{??????????????},Third Edition, ???????,2008.7
	
	\bibitem[WATSON93a]{WATSON93a}
	WATSON, B. W. \textit{A taxonomy of finite automata construction algorithms}, Computing Science Note 93/43, Eindhoven University of Technology, The Netherlands, 1993. Available by ftp from ftp.win.tue.nl in pub/techreports/pi.
	
	\bibitem[WATSON93b]{WATSON93b}
	WATSON, B. W. \textit{A taxonomy of finite automata minimization algorithms}, Computing Science Note 93/44, Eindhoven University of Technology, The Netherlands, 1993. Available by ftp from ftp.win.tue.nl in pub/techreports/pi.
	
	\bibitem[WATSON94a]{WATSON94a}
	WATSON, B. W. \textit{An introduction to the FIRE engine: A C++ toolkit for FInite automata and Regular Expressions}, Computing Science Note 94/21, Eindhoven University of Technology, The Netherlands, 1994. Available by ftp from ftp.win.tue.nl in pub/techreports/pi
	
	\bibitem[WATSON94b]{WATSON94b}
	WATSON, B.W. \textit{The design. and implementation of the FIRE engine:	A C++ toolkit for FInite automata and Regular Expressions}, Computing Science Note 94/22, Eindhoven University of Technology, The Netherlands, 1994. Available by ftp from ftp.win.tue.nl in pub/techreports/pi.
	
	\bibitem[Chrison2007]{Chrison2007}
	Christos G. Cassandras and St$\acute{e}$phane Lafortune, \textit{Introduction to Discrete Event Systems},Second Edition,New York,Springer,2007
	
	\bibitem[Wonham2018]{Wonham2018}
	W. M. Wonham and Kai Cai,\textit{Supervisory Control of Discrete-Event Systems}, Revised 2018.01.01
	
	\bibitem[Jean2018]{Jean2018}
	Jean-$\acute{E}$ric Pin, \textit{Mathematical Foundations of Automata Theory},Version of June 15,2018
	
	\bibitem[???2013]{???2013}
	???,???, \textit{????????????3??}, ???????,2013.05
	
	
	\bibitem{Lipschutz2007}
	S. Lipschutz and M. L. Lipson, \textit{Schaum's Outline of Theory and Problems of Discrete Mathematics}, Third Edition, New York: McGraw-Hill, 2007.
	
	\bibitem{Rosen2007}
	K. H. Rosen, \textit{Discrete Mathematics and Its Applications}, Seventh Edition, New York: McGraw-Hill, 2007.
	
	\bibitem{Maclane1988}
	S. Maclane and G. Birkhoff, \textit{Algebra}, Third Edition, New York: Bchelsea Publishing Company, 1988.
\end{thebibliography}


\begin{thebibliography}{99}
	
	
	\bibitem{Lipschutz2007}
	S. Lipschutz and M. L. Lipson, \textit{Schaum's Outline of Theory and Problems of Discrete Mathematics}, Third Edition, New York: McGraw-Hill, 2007.
	
	\bibitem{Rosen2007}
	K. H. Rosen, \textit{Discrete Mathematics and Its Applications}, Seventh Edition, New York: McGraw-Hill, 2007.
	
	\bibitem{Maclane1988}
	S. Maclane and G. Birkhoff, \textit{Algebra}, Third Edition, New York: Bchelsea Publishing Company, 1988.
\end{thebibliography}
