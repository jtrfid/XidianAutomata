\chapter{\cite{蒋宗礼2013}(第1章 绪论)}

\begin{itemize}
	\item 集合:集合的表示、集合之间的关系、集合的基本运算。
	\item 关系:主要介绍了二元关系相关的内容。包括等价关系、等价分类、关系合成、关系闭包。
	\item 递归定义与归纳证明。	
	\item 图:无向图、有向图、树的基本概念。
	\item 语言与形式语言:自然语言的描述,形式语言和自动机理论的出现,形式语言和自动机理论对计算机科学与技术学科人才能力培养的作用
	\item 基本概念:字母表、字母、句子、字母表上的语言、语言的基本运算
	
\end{itemize}

\section{集合的基础知识}
\subsection{集合之间的关系}
\begin{definition}
	设$A,B$是两个集合,如果集合$A$中的元素都是集合$B$的元素,则称集合$A$是集合$B$的\emph{子集(subset)},集合$B$是集合$A$的\emph{包集(container)}。记作$A\subseteq B$,也可以记作$B\supseteq A$。
\end{definition}

由定义可知,$A\subseteq B$的充要条件是:对于A中的每一个元素$a$,均有$a\in B$。为了简洁起见,$P_1$是$P_2$的充要条件记为
$$P_1\iff P_2$$
或者
$$P_2\text{ iff }P_1$$

经常使用全称量词和存在量词:$\forall x$表示对所有的$x$; $\exists x$表示存在一个$x$。
按照此约定,有

$A\subseteq B\iff\forall x\in A,x\in B$成立,\\
也就是

$A\subseteq B\text{ iff }\forall x\in A,x\in B$成立。

\begin{definition}
	设$A,B$是两个集合,如果$A\subseteq B,\text{且}\exists x\in B,\text{但}x\notin B$,则称$A$是$B$的\emph{真子集(proper subset)},记作$A\subset B$。
\end{definition}

\begin{definition}
	如果集合$A,B$含有的元素完全相同,则称集合$A$与集合$B$\emph{相等(equivalence)},记作$A=B$。
\end{definition}

对任意集合$A,B,C$,不难得出下列结论:
\begin{enumerate}
	\item $A=B$ iff $A\subseteq B$且$B\subseteq A$。
	\item 如果$A\subseteq B$,则$|A|\le|B|$。
	\item 如果$A\subset B$,则$|A|\le|B|$。
	\item 如果A是有穷集,且$A\supset B$,则$|B|>|A|$。
	\item 如果$A\subseteq B$,则$\forall x\in A$,有$x\in B$。
	\item 如果$A\subset B$,则$\forall x\in A$,有$x\in B$并且$\exists x\in B$,但$x\notin A$。
	\item 如果$A\subseteq B$且$B\subseteq C$,则$A\subseteq C$。
	\item 如果$A\subset B$且$B\subset C$,或者$A\subseteq B$且$B\subset C$,或者$A\subset B$且$B\subseteq C$,则$A\subset C$。
	\item 如果$A=B$,则$|A|=|B|$。
\end{enumerate} 

\begin{definition}
	设$A,B$是两个集合,$A$与$B$的\emph{对称差(symmetric difference)}是一个集合,该集合由属于$A$但不属于$B$,以及属于$B$但不属于$A$的所有元素组成,记作$A\otimes B$。
	$$A\otimes B = \{a|a\in A\text{且}a\notin B\text{或者}a\notin B\text{且}a\in B\}$$
\end{definition}

显然,对集合$A,B$,有
$$A\otimes B = (A\cup B) - (A\cap B) = (A-B)\cup(B-A)$$

$\otimes$为对称差运算符,$A\otimes B$读作$A$对称减$B$($A$与$B$的对称差)。

\begin{definition}
	设$A,B$是两个集合,$A$与$B$的\emph{笛卡儿积(Cartesian product)}是一个集合,该集合该集合是由所有这样的有序对$(a,b)$组成的:其中$a\in A,b\in B$ ,记作$A\times B$。
	$$A\times B = \{(a,b)|a\in A \& b\in B\}$$
\end{definition}
$\times$为笛卡儿乘运算符。$A\times B$读作$A$叉乘$B$。

对任意集合$A,B,C$,不难得出下列结论:
\begin{enumerate}
	\item $A\times B\ne B\times A$。
	\item $(A\times B)\times C\ne A\times(B\times C)$。
	\item $A\times A\ne A$。
	\item $A\times\emptyset=\emptyset$。
	\item $A\times(B\cup C)=(A\times B)\cup(A\times C)$。
	\item $(B\cup C)\times A=(B\times A)\cup(C\times A)$。
	\item $A\times(B\cap C)=(A\times B)\cap(A\times C)$。
	\item $(B\cap C)\times A=(B\times A)\cap(C\times A)$。
	\item $A\times(B-C)=(A\times B)-(A\times C)$。
	\item $(B-C)\times A=(B\times A)-(C\times A)$。
	\item 当$A,B$为有穷集时,$|A\times B|=|A||B|$。
\end{enumerate} 

\begin{definition}
	设$A$是一个集合,$A$的\emph{幂集(power set)}是一个集合,该集合由$A$的所有子集组成,记作$2^A$。
	$$2^A=\{B|B\subseteq A\}$$
\end{definition}
$2^A$读作$A$的幂集。

对于任意集合$A,B$,则有下列结论:
\begin{enumerate}
	\item $\emptyset\in 2^A$。
	\item $\emptyset\subseteq 2^A$。
	\item $\emptyset\subset 2^2$。
	\item $2^\emptyset=\{\emptyset\}$。
	\item $A\in 2^A$。
	\item 如果$A$是有穷集合,则$|2^A|=2^|A|$。
	\item $2^{A\cap B} = 2^A\cap 2^B$。
	\item if $A\subseteq B$, then $2^A\subseteq 2^B$。
\end{enumerate}


\begin{definition}
	设$A$是论域$U$上的一个集合,$A$的\emph{补集(complementary set)}是一个集合,该集合由在$U$中,但不在$A$中的所有元素组成,记作$\bar{A}$。
	$$\bar{A}=U-A$$
\end{definition}
$\bar{A}$读作$A$(关于论域$U$)的补集($U$中子集$A$的补集)。

在实际工作和生活中,人们都会在一定的范围内讨论问题,讨论问题的这个范围叫作论域。如果集合$A$是论域$U$上的一个集合,则$A\subseteq U$。

设$U$是论域,$A,B$是$U$上的集合,则有下列结论:
\begin{enumerate}
	\item $\bar{\emptyset}=U$。
	\item $\bar{U}=\emptyset$。
	\item if $A\subseteq B$, then $\bar{B}\subseteq \bar{A}$。
	\item $A\cup \bar{A}=U$。
	\item $A\cap \bar{A}=\emptyset$。
	\item $B=\bar{A}\iff A\cup B=U\& A\cap B=\emptyset$。
	\item De Morgan定律\\ 
	$\overline{A\cap B}=\bar{A}\cup\bar{B}$。\\
	$\overline{A\cup B}=\bar{A}\cap\bar{B}$。
\end{enumerate}


\section{关系}
\begin{itemize}
	\item 二元关系 
	\item 递归定义与归纳证明 
	\item 关系的闭包 
\end{itemize}

\subsection{二元关系}

\begin{definition}
	设$A,B$是两个集合,任意的$R\subset A\times B,R$是$A$到$B$的\emph{二元关系(binary relation)}。
	
	$(a,b)\in R$,表示$a$与$b$满足关系$R$,按照中缀形式,也可以表示为$aRb$。
	
	$A$称为\emph{定义域(domain)}, $B$称为\emph{值域(range)}。
	
	当$A=B$时,则称$R$是$A$上的二元关系。
\end{definition}

\begin{definition}
	$R$是$A$上的二元关系,
	\begin{enumerate}
		\item 如果对任意一个$a\in A$,有$(a,a)\in R$,则称$R$是\emph{自反的(reflexive)}。
		\item 如果对任意一个$a\in A$,有$(a,a)\notin R$,则称$R$是\emph{反自反的(irreflexive)}。
		\item 如果对任意的$a,b\in A$,当$(b,a)\in R$时,必有$(a,b)\in R$,
		则称$R$是\emph{对称的(symmetric)}。
		\item 如果对任意的$a,b\in A$,当$(b,a)\in R$和$(a,b)\in R$同时成立,必有$a=b$,
		则称$R$是\emph{反对称的(asymmetric)}。
		\item 如果对任意的$a,b,c\in A$,当$(a,b)\in R$和$(b,c)\in R$同时成立,必有$(a,c)\in R$,
		则称$R$是\emph{传递的(transition)}。
	\end{enumerate}
    条件(1),(3),(5)合并在一起,叫作关系的三歧性:自反性,对称性,传递性。
\end{definition}

\begin{example}
	关系的性质。
	\begin{enumerate}
		\item $=$关系是自反的,对称的,传递的。
		\item $>,<$关系是反自反的,传递的。
		\item $\ge,\le$关系是自反的,反对称的,传递的。
		\item 集合之间的包含关系是自反的,反对称的,传递的。
		\item 整数集上的模$n$同余关系是自反的,对称的,传递的。
		\item 通常意义下的父子关系是反自反的,非传递的。
		\item 通常意义下的兄弟关系是反自反的,传递的。
		\item 通常意义下的祖先关系是反自反的,传递的。
	\end{enumerate}
\end{example}

\subsection{等价关系与等价类}

\begin{definition}
	如果集合$A$上的二元关系$R$是自反的,对称的,传递的,则称$R$是\emph{等价关系(equivalence relation)}。
\end{definition}

\begin{example}
	等价关系示例。
	\begin{enumerate}
		\item 实数集上的“=”关系。
		\item 整数集上的模$n$同余关系。
		\item 通常意义下的“在同一个学校工作”的关系。
		\item “户口在同一省、市、自治区”的关系。
	\end{enumerate}
    由此,可以考虑利用集合$S$上的等价关系$R$将$S$划分成若干个等价类。
\end{example}

\begin{definition}
	设$R$是集合$S$上的等价关系,则满足如下要求的$S$的划分$S_1,S_2,S_3,\cdots,S_n,\cdots$称为$S$关于$R$的等价划分,$S_i$称为\emph{等价类(equivalence class)}。它们满足以下各条:
	\begin{enumerate}
		\item $S=S_1\cup S_2\cup S_3\cup\cdots\cup S_n\cup\cdots$。
		\item if $i\ne j$, then $S_i\cap S_j=\emptyset$。
		\item 对任意的$i,S_i$中的任意两个元素$a,b,aRb$恒成立。
		\item 对任意的$i,j,i\ne j,S_i$中的任意元素$a$和$S_j$中的任意元素$b,aRb$恒不成立。
	\end{enumerate}

    $R$将$S$分成的等价类的个数称为$R$在$S$上的\emph{指数(index)}。有时候,$R$将$S$分成有穷多个等价类,此时称$R$具有有穷指数,反之称为无穷指数。
\end{definition}

\begin{example} 等价类。
	\begin{enumerate}
		\item "="关系将自然数$\mathbb{N}$分成无穷多个等价类:$\{1\},\{2\},\{3\},\cdots$。
		\item 非负整数集上的模5同余关系将$\{0,1,2,3,\cdots\}$分成5个等价类:\\
		$\{0,5,10,15,20,\cdots\}$\\
		$\{1,6,11,16,21,\cdots\}$\\
		$\{2,7,12,17,22,\cdots\}$\\
		$\{3,8,13,18,23,\cdots\}$\\
		$\{4,9,14,19,24,\cdots\}$
		\item 某计算机学院2001年招收本科生420名,分成12个班,按同班同学关系划分,这420名学生分成12个等价类,每个等价类对应一个班。
	\end{enumerate}
\end{example}

\begin{note}
值得注意的是,给定集合$S$上的一个等价关系$R$,$R$就确定了$S$的一个等价划分。当给定另一个不同的等价关系时,它会确定$S$的一个新的等价划分。
\end{note}

\begin{example}令$S=\{1,2,3,4\}$.
	\begin{enumerate}
		\item 通常意义的"="关系将$S$分成4个等价类:$\{1\},\{2\},\{3\},\{4\}$。
		\item 如果取$R=\{(1,1),(2,1),(1,2),(2,2),(3,3),(3,4),(4,3),(4,4)\}$,则将$S$分成两个等价类:$\{1,2\},\{3,4\}$。
		
		\begin{tikzpicture}[>=latex, shorten >=1pt,node distance=0.7in, on grid, auto]
		\node[state] (q1) {$1$};
		\node[state] (q2) [below of=q1] {$2$};
		\node[state] (q3) [below of=q2] {$3$};
		\node[state] (q4) [below of=q3] {$4$};
		\node[state] (q11) [right of=q1]{$1$};
		\node[state] (q22) [below of=q11] {$2$};
		\node[state] (q33) [below of=q22] {$3$};
		\node[state] (q44) [below of=q33] {$4$};
		\path[->]
		(q1) edge node {} (q11)
		(q2) edge node {} (q11)
		(q1) edge node {} (q22)
		(q2) edge node {} (q22)
		(q3) edge node {} (q33)
		(q3) edge node {} (q44)
		(q4) edge node {} (q33)
		(q4) edge node {} (q44)
		;
		\end{tikzpicture}
	\end{enumerate}
\end{example}

\subsection{关系的合成(composition)}

\begin{definition}
	设$R_1\subseteq A\times B$是$A$到$B$的关系,$R_2\subseteq B\times C$是$B$到$C$的关系,则$R_1$与$R_2$的\emph{合成(composition)} $R_1\circ R_2$是$A$到$C$的关系。
	$$R_1\circ R_2=\{(a,c)|\exists(a,b)\in R_1\&(b,c)\in R_2\}$$
\end{definition}

\begin{example}
	设$R_1,R_2$是集合$\{1,2,3,4\}$上的关系,其中
	
	$R_1={(1,1),(1,2),(2,3),(3,4)}$
	
	$R_1={(2,4),(4,1),(4,3),(3,1)}$\\
	则
	
	$R_1\circ R_2=\{(1,4),(2,1),(3,1),(3,3)\}$
	
	\begin{tikzpicture}[>=latex, shorten >=1pt,node distance=0.7in, on grid, auto]
	\node[state] (a1) {$a1$};
	\node[state] (a2) [below of=a1] {$a2$};
	\node[state] (a3) [below of=a2] {$a3$};
	\node[state] (a4) [below of=a3] {$a4$};
	\node[state] (b1) [right of=a1] {$b1$};
	\node[state] (b2) [below of=b1] {$b2$};
	\node[state] (b3) [below of=b2] {$b3$};
	\node[state] (b4) [below of=b3] {$b4$};
	\node[state] (c1) [right of=b1] {$c1$};
	\node[state] (c2) [below of=c1] {$c2$};
	\node[state] (c3) [below of=c2] {$c3$};
	\node[state] (c4) [below of=c3] {$c4$};
	\path[->]
	(a1) edge node {} (b1)
	(a1) edge node {} (b2)
	(a2) edge node {} (b3)
	(a3) edge node {} (b4)
	(b2) edge node {} (c4)
	(b4) edge node {} (c1)
	(b4) edge node {} (c3)
	(b3) edge node {} (c1)
	;
	\end{tikzpicture}
\end{example}

设$R_1,R_2,R_3$分别是$S$上的二元关系,可以证明以下结论:
\begin{enumerate}
	\item $R_1\circ R_2\ne R_2\circ R_1$。
	\item $(R_1\circ R_2)\circ R_3 = R_1\circ (R_2\circ R_3)$。(结合律)
	\item $(R_1\cup R_2)\circ R_3 = (R_1\circ R_3)\cup(R_2\circ R_3)$。(合成对$\cup$的右分配律)
	\item $R_3\circ(R_1\cup R_2) = (R_3\circ R_1)\cup(R_3\circ R_2)$。(合成对$\cup$的左分配律)
	\item $(R_1\cap R_2)\circ R_3\subseteq(R_1\circ R_3)\cap(R_2\circ R_3)$。(合成对$\cap$的右分配律)
	\item $R_3\circ(R_1\cap R_2)\subseteq(R_3\circ R_1)\cap(R_3\circ R_2)$。(合成对$\cap$的左分配律)
\end{enumerate}

\subsection{递归定义(recurisive definition)与归纳证明}
\begin{itemize}
	\item 递归定义(recurisive definition)
	\begin{itemize}
		\item 又称为归纳定义(indutive definition),它来定义一个集合。
		\item 集合的递归定义由三个部分组成:
		\begin{itemize}
			\item 基础(basis): 用来定义该集合的最基本的元素。
			\item 归纳(induction): 指出用集合中的元素来构造集合的新元素的规则。
			\item 极小性限定: 指出一个对象是所定义集合中的元素的充要条件是它可以通过有限次的使用基础和归纳条款中所给的规定构造出来。
		\end{itemize}
	\end{itemize}
    \item 归纳证明
    \begin{itemize}
    	\item 与递归定义相对应
    	\item 归纳证明方法包括三大步:
    	\begin{itemize}
    		\item 基础(basis): 证明最基本元素具有相应性质。
    		\item 归纳(induction): 证明如果某些元素具有相应性质,则根据这些元素用所规定的方法得到新元素也有相应的性质。
    		\item 根据归纳法原理,所有的元素具有相应的性质。
    	\end{itemize}
    \end{itemize}
\end{itemize}

\begin{example} \label{exr} 算术表达式的递归定义:
	\begin{enumerate}
		\item 基础(basis): 常数是算术表达式,变量是算术表达式;
		\item 归纳(induction): 如果$E_1,E_2$是算术表达式,则$+E_1,-E_1,E_1+E_2,E_1-E_2,E_1\ast E_2,E_1/E_2,E_1\land {E_2},Fun(E_1)$是算术表达式。其中$Fun$为函数名。
		\item 只有满足(1)和(2)的式子才是算术表达式。
	\end{enumerate}
\end{example}

\begin{definition}
	设$R$是$S$上的关系,我们递归地定义$R^n$的幂:
	\begin{enumerate}
		\item $R^0 = \{(a,a)|a\in S\}$
		\item $R^i = R^{i-1}R\quad (i=1,2,3,\cdots)$
	\end{enumerate}
\end{definition}

\begin{example}
对有穷集合$A$, 证明$|2^A| = 2^{|A|}$。
\begin{proof}
	设$A$为一个有穷集合,施归纳于$|A|$:
	\begin{enumerate}
		\item 基础(basis): 当$|A|=0$时,由幂集定义,$|2^A|=|\{\emptyset \}|=1$。而$2^{|A|}=2^0=1$。所以有$|2^A|=2^{|A|}$对$|A|=0$成立。
		\item 归纳(induction): 假设$|A|=n$时结论成立,这里$n\ge 0$,往证当$|A|=n+1$时结论成立。\\
		为此,不妨设$A=B\cup \{a\},a\notin B$,即\\
		$|2^A|=B\cup \{a\}=|B|+|\{a\}|=|B|+1$\\
		由幂集的定义知\\
		$2^A = 2^B \cup \{C\cup \{ a\} | C\in 2^B\}$\\
		由于$a\notin B$,所以\\
		$2^B\cap \{C\cup \{a\}|C\in 2^B \} = \emptyset$\\
		由$\{C\cup \{a\}|C\in 2^B \}$的构造方法知道,可以按如下方法构造一个一一对应的映射$f:\{C\cup \{a\}|C\in 2^B \}\to 2^B$,使\\
		$f(C\cup\{a\})=C$\\
		所以\\
		$|\{C\cup \{a\}|C\in 2^B \}|=|2^B|$\\
		故
		\begin{align*}
			|2^A|&=|2^B \cup \{C\cup \{ a\} | C\in 2^B\}|\\
			&=|2^B|+|\{C\cup \{a\}|C\in 2^B \}| \\
			&=|2^B|+|2^B| \\
			&=2|2^B| 
	    \end{align*}
		显然,$|B|=n$,由归纳假设知\\
		$|2^B|=2^{|B|}$\\
		从而有\\
		$|2^A|=2|2^B|=2\times|2^B|=2^{|B|+1}=2^{|A|}$\\
		这就是说,结论对$|A|=n+1$成立。
		\item 由归纳法原理,结论对任意有穷集合成立。
	\end{enumerate}
\end{proof}	
\end{example}

\begin{example}
	表达式的前缀形式是指将运算符写在前面,后跟相应的运算对象。如:$+E_1$的前缀形式为$+E_1$,$E_1+E_2$的前缀形式为$+E_1E_2$,$E_1\ast E_2$的前缀形式为$\ast E_1E_2$, $E_1\land E_2$的前缀形式为$\land E_1E_2$,$Fun(E_1)$ 的前缀形式为$FunE_1$。
	证明\textit{Example \ref{exr}}所定义的表达式可以用这里定义的前缀形式表示。 	
\end{example}

\begin{proof}
	设$E$为\textit{Example \ref{exr}}所定义的算术表达式,现对$E$中所含的运算符(包括函数引用)的个数实施归纳。设$E$中含$n$个运算符。
	\begin{enumerate}
		\item 基础(basis): 当$n=0$时,表达式为一个常数或者变量,结论显然成立。
		\item 归纳(induction): 假设$n\le k$时结论成立,这里$k\ge 0$,往证当$n=k+1$时结论成立。\\
		由于$E$中含有$k+1$个运算符,所以必须是如下情况中的一种:
		\begin{enumerate}
			\item 当$E=+E_1$时,我们知道$E_1$中的运算符个数为$k$,由归纳假设,$E_1$有对应的前缀形式$F_1$,从而$E$的前缀形式为$+F_1$。
			\item 当$E=-E_1$时,类似地,$E$的前缀形式为$-F_1$。
			\item 当$E=E_1+E_2$时,$E_1,E_2$中的运算符个数分别小于等于$k$,由归纳假设,$E_1,E_2$有对应的前缀形式$F_1,F_2$,从而$E$的前缀形式为$+F_1F_2$。
			\item 对$E=E_1-E_2,E=E_1\ast E_2,E=E_1/E_2,E=E_1\land E_2 $的情况进行类似的讨论,它们的前缀形式分别为: $-F_1F_2,\ast E_1E_2,/E_1E_2,\land E_1E_2$。
			\item 当$E=Fun(E_1)$时,我们知道$E_1$中的运算符个数为$k$,由归纳假设,$E_1$有对应的前缀形式$F_1$,从而$E$的前缀形式为$FunF_1$。
		\end{enumerate}
	    综上所述,结论对$n=k+1$成立。
	    \item 由归纳法原理,结论对\textit{Example \ref{exr}}所定义的所有表达式成立。
	\end{enumerate}
\end{proof}

\subsection{关系的闭包}

\begin{definition}
	设$P$是关于关系的性质的集合,关系$R$的\emph{$P$闭包(closure)}是包含$R$并且具有$P$中所有性质的最小关系。
\end{definition}

\begin{definition}\label{def:R+}
	设$R$是$S$上二元关系,$R$的\emph{正闭包(positive closure)}$R^+$的定义为:
	\subitem(1) $R\subseteq R^{+}$。
	\subitem(2) 如果$(a,b),(b,c) \in R^{+}$,则$(a,c)\in R^{+}$。
	\subitem(3) 除(1)、(2)外,$R^{+}$不包含有其他任何元素。 
    
    可以证明,$R^+$具有传递性,因此又称其为\emph{传递闭包(transitive closure)}。还可以证明,对于任意的二元关系$R$,有
    $$R^{+} = R\cup R^2 \cup R^3 \cup R^4 \cup \dots$$
    且当$S$为有穷集时,有
    \[R^{+} = R\cup R^2 \cup R^3 \cup \cdots \cup R^{|S|} \]
\end{definition}

\begin{definition}\label{def:R*}
	设$R$是$S$上二元关系,$R$的\emph{克林闭包(Kleene closure)}$R^\ast$的定义为:
	\subitem(1) $R^0 \subseteq R^*,R\subseteq R^{\ast}$.
	\subitem(2) 如果$(a,b),(b,c)\in R^{\ast}$则$(a,c)\in R^{\ast}$.
	\subitem(3) 除(1),(2)外,$R^{\ast}$不再含有其他任何元素.
	
	可以证明,$R^\ast$具有自反性和传递性,因此又称其为\emph{自反传递闭包(reflexive and transitive closure)}。
\end{definition}

由定义\ref{def:R+}和\ref{def:R*}可知,对于任意二元关系$R$,有
\begin{align*}
	R^{\ast} &=R^{0}\cup R^{+} \\
	&=R^{0}\cup R\cup R^{2}\cup R^{3}\cup \dots
\end{align*}
而且当$S$为有穷集时:
\[R^{\ast} =R^{0}\cup R\cup R^{2}\cup R^{3}\cup \dots \cup R^{|S|} \]

设$R_1,R_2$是$S$上的两个二元关系,则
\subitem(1) $\emptyset^{+}=\emptyset$
\subitem(2) $(R_1^{+})^{+} = R_1^{+}$
\subitem(3) $(R_1^{\ast})^{\ast} = R_1^{\ast}$
\subitem(4) $R_1^{+}\cup R_2^{+} \subseteq (R_1 \cup R_2)^{+}$
\subitem(5) $R_1^{\ast}\cup R_2^{\ast} \subseteq (R_1\cup R^2)^{\ast}$	

\section{语言}
\subsection{字母表(alphabet)}
\begin{itemize}
	\item \emph{字母表}是一个非空有穷集合,字母表中的元素称为该字母表的一个\emph{字母(letter)}。又叫做\emph{符号(symbol)}、或者\emph{字符(character)}。
	\item 非空性
	\item 有穷性
	\item 字符的两个特性
		\subitem{-} 整体性(monolith), 也叫不可分性
		\subitem{-} 可辨认性(distinguishable),也叫可去区分性
	\item 字母表的乘积(product)
	\[\Sigma_1\Sigma_2 = \{ab|a\in \Sigma_1,b\in \Sigma_2 \} \]
	\item 字母表$\Sigma$的n次幂
		\subitem $\Sigma^0 = \{\epsilon\}$
		\subitem $\Sigma^n = \Sigma^{n-1}\Sigma$
		\subitem $\epsilon$是由$\Sigma$中的0个字符组成的。
	\item $\Sigma$的\emph{正闭包}
		\[\Sigma^{+} = \Sigma \cup \Sigma^2 \cup \Sigma^3 \cup \cdots \]
		\[\Sigma^+ = \{x|x\text{是}\Sigma\text{中的至少一个字符连接而成的字符串}\} \]
	\item $\Sigma$的\emph{克林闭包}
		\[\Sigma^{\ast} = \Sigma^0 \cup \Sigma^+ = \Sigma^0 \cup \Sigma \cup \Sigma^2 \cup \Sigma^3 \cup \cdots \]
		\[\Sigma^{\ast} = \{x|x\text{是}\Sigma\text{中的若干个,包括0个字符,连接而成的字符串}\} \]
\end{itemize}

\begin{example} \{alphabet\}
	\begin{flushleft}
		\{a,b,c,d\}\\
		\{a,b,c,\dots,z\}\\
		\{0,1\} \\
		\{a,$a^{\prime}$,b,$b^{\prime}$ \}\\
		\{aa,ab,bb \}\\
		$\{\infty,\land,\lor,\geq,\leq \}$
	\end{flushleft}
\end{example}

\begin{example} product
	\begin{flushleft}
		\{0,1\}\{0,1\} = \{00,01,10,00\}\\
		\{0,1\}\{a,b,c,d\} = \{0a,0b,0c,0d,1a,1b,1c,1d\}\\
		\{a,b,c,d\}\{0,1\} = \{a0,a1,b0,b1,c0,c1,d0,d1\}\\
		\{aa,ab,bb\}\{0,1\} = \{aa0,aa1,ab0,ab1,bb0,bb1\}
	\end{flushleft}
\end{example}

\begin{example} $\Sigma^0,\Sigma^{\ast}$ 
	\begin{align*}
	\{0,1\}^+ &= \{0,1,00,01,11,000,001,010,011,100,\dots\}\\
	\{0,1\}^{\ast} &= \{\epsilon,0,1,00,01,11,000,001,010,011,100,\dots\}\\
	\{a,b,c,d\}^+ &= \{a,b,c,d,aa,ab,ac,ad,ba,bb,bc,bd,\dots,aaa,aab,aac,aad,aba,abb,abc,\dots\}\\
	\{a,b,c,d\}^{\ast} &= \{\epsilon,a,b,c,d,aa,ab,ac,ad,ba,bb,bc,bd,\dots,aaa,aab,aac,aad,aba,abb,abc,\dots\}
	\end{align*}
\end{example}

\subsection{句子(sentence)/字(word)/字符串(string)}
\begin{itemize}
	\item 别称\\
		句子(sentence),(字符、符号)行(line),(字符、符号)串(string).
	\item 句子(sentence)\\
		$\Sigma$是一个字母表,$\forall \in \Sigma^{\ast},x$叫做$\Sigma$上的一个句子。
	\item 句子相等\\
		两个句子被认为相等的,如果它们对应位置上的字符都对应相等。
	\item 句子的长度(length)
		\subitem{-} $\forall x\in \Sigma^{\ast}$,句子$x$中字符出现的总个数叫做该句子的长度,记作$|x|$。
		\subitem{-} 长度为0的字符串叫\emph{空句子},记作$\epsilon$
	\item 串$x$的$n$次幂
	\begin{align*}
		x^0 &= \epsilon \\
		x^n &= x^{n-1}x
	\end{align*}
\end{itemize}

\begin{note} 注意事项
	\begin{itemize}
		\item $\epsilon$是一个句子
		\item $\{\epsilon\}\ne \emptyset$。这是因为$\{\epsilon\}$不是一个空集,它是含有一个空句子的$\epsilon$的集合。$|\{\epsilon\}|=1,|\emptyset|=0$
	\end{itemize}
\end{note}

\begin{example}
	$$|abaabb|=6$$
	$$|bbaa|=4$$
	$$|\epsilon|=0$$
\end{example}

\begin{example}
	x=001,y=1101
	\begin{align*}
	x^0 &= y^0 = \epsilon \\
	x^4 &= 001001001001 \\
	y^4 &= 1101110111011101
	\end{align*}
\end{example}

\subsection{并置/连结(concatenation)}
\begin{itemize}
	\item 并置/连结(concatenation)
	\subitem{-} $x,y\in \Sigma^{\ast},x,y$的并置是由串$x$直接相接串$y$组成的。记作$xy$.
	\item $\Sigma^{\ast}$上的并置运算性质
	\begin{enumerate}
		\item 结合律: $(xy)z=x(yz)$
		\item 左消去律: if $xy=xz$,then $y=z$
		\item 右消去律: if $yx=zx$,then $y=z$
		\item 惟一分解性: 存在惟一确定的$a_1,a_2,\dots,a_n \in \Sigma$,使得$x=a_1a_2\cdots a_n$.
		\item 单位元素: $\epsilon x=x\epsilon=x$
	\end{enumerate} 
\end{itemize}

\subsection{前缀与后缀}
设$x,y,z,w,v\in \Sigma^{\ast},$且$x=yz,w=yv$
\begin{enumerate}
	\item $y$是$x$的前缀(prefix)
	\item 如果$z\ne \epsilon$,则$y$是x的真前缀(proper prefix).
	\item $z$是$x$的后缀(suffix)
	\item 如果$y\ne \epsilon$,则$z$是$x$的真后缀(proper suffix)
	\item $y$是$x$和$w$的公共前缀(common prefix)
	\item 如果$x$和$w$的任何公共前缀都是$y$的前缀, 则$y$是$x$和$w$的最大公共前缀。
	\item 如果$x=zy$和$w=vy$, 则$y$是$x$和$w$的公共后缀(common suffix)。
	\item 如果$x$和$w$的任何公共后缀都是$y$的后缀, 则$y$是$x$和$w$的最大公共后缀。
\end{enumerate}

\begin{example}
	$\Sigma = \{a,b\}$上的句子abaabb:\\
	前缀: $\epsilon,a,ab,aba,abaa,abaab,abaabb$ \\
	真前缀: $\epsilon,a,ab,aba,abaa,abaab$ \\
	后缀: $\epsilon,b,bb,abb,aabb,baabb,abaabb$ \\
	真后缀: $\epsilon,b,bb,abb,aabb,baabb$
\end{example}

\noindent \textbf{结论}

\begin{enumerate}
	\item $x$的任意前缀$y$有惟一的一个后缀$z$与之对应,使得$x=yz$;反之亦然。
	\item $x$的任意真前缀$y$有惟一的一个真后缀$z$与之对应,使得$x=yz$;反之亦然。
	\item $|\{w|w\text{是}x\text{的后缀}\}| = |\{w|w\text{是}x\text{的前缀}\}|$
	\item $|\{w|w\text{是}x\text{的真后缀}\}| = |\{w|w\text{是}x\text{的真前缀}\}|$
	\item $\{w|w\text{是}x\text{的前缀}\}| = |\{w|w\text{是}x\text{的真前缀}\cup \{x\}\}$
	\item $|\{w|w\text{是}x\text{的前缀}\}| = |\{w|w\text{是}x\text{的真前缀} + 1\}|$
	\item $\{w|w\text{是}x\text{的后缀}\}| = |\{w|w\text{是}x\text{的真后缀}\cup \{x\}\}$
	\item $|\{w|w\text{是}x\text{的后缀}\}| = |\{w|w\text{是}x\text{的真后缀} + 1\}|$
	\item 对于任意字符串$w,w$是自身的前缀,但不是自身的真前缀; $w$是自身的后缀,但不是自身的真后缀。
	\item 对于任意字符串$w,\epsilon$是$w$的前缀,且是$w$的真前缀; $\epsilon$是$w$的后缀,且是$w$的真后缀。
\end{enumerate}

\noindent \textbf{约定}
\begin{itemize}
	\item 用小写字母表中较为靠前的字母$a,b,c,\dots$表示字母表中的字母
	\item 用小写字母表中较为靠后的字母$x,y,z,\dots$表示字母表中的句子(字)
	\item 用$x^T$表示$x$的倒序。例如,如果$x=abc$,则$X^T=cba$
\end{itemize}

\subsection{子串(substring)}
\begin{itemize}
	\item 子串(substring)
		\subitem{-} $w,x,y,z\in \Sigma^{\ast}$,且$w=xyz$,则称$y$是$w$的子串。
	\item 公共子串(common substring)
		\subitem{-} $t,u,v,w,x,y,z\in \Sigma^{\ast}$,且$t=uyv,w=xyz$,则称$y$是$t$和$w$的公共子串(common substring)。如果$y_1,y_2,\dots,y_n$是$t$和$w$的公共子串,且$max\{|y_1|,|y_2|,\dots,|y_n|\}=|y_j|$,则称$y_j$是$t$和$w$的\emph{最大公共子串}。
		\subitem{-} 两个串的最大公共子串并不一定是惟一的。
\end{itemize}

\subsection{语言(language)}
$\forall \in\Sigma^{\ast},L$称为字母表$\Sigma$上的一个\emph{语言(language)},$\forall x\in L,x$叫做$L$的一个句子(sentence)/字(word)/字符串(string)。
\begin{example}
	$\Sigma = \{0,1\}$上的不同语言
	\{00,11\}\\
	\{0,1\}\\
	\{0,1,00,11\}\\
	\{0,1,00,11,01,10\}\\
	\{00,11\}$^{\ast}$\\
	\{01,10\}$^{\ast}$\\
	\{00,01,10,11\}$^{\ast}$\\
	\{0\}\{0,1\}$^{\ast}$\{1\}\\
	\{0,1\}$^{\ast}$\{111\}\{0,1\}$^{\ast}$\\
\end{example}

\subsection{语言的\emph{乘积(product)}}
$L_1\subseteq \Sigma_1^{\ast},L_2\subseteq\Sigma_2^{\ast}$,语言$L_1$与$L_2$的\emph{乘积(product)}是一个语言,该语言定义为:
\[L_1L_2=\{xy|x\in L_1,y\in L_2\}\]
是字母表$\Sigma_1\cup\Sigma_2$上的语言。
\begin{example} $\Sigma = \{0,1\}$
	\begin{align*}
	L_1 &=\{0,1\}\\
	L_2 &=\{00,01,10,11\} \\
	L_3 &=\{0,1,00,01,10,11,000,\dots\} =\Sigma^{+} \\
	L_4 &=\{\epsilon,0,1,00,01,10,11,000,\dots\}=\Sigma^{\ast} \\
	L_5 &=\{0^n|n\geq 1\} \\
	L_6 &=\{0^n1^n\geq 1\} \\
	L_7 &=\{1^n|n\geq 1\} \\
	L_8 &=\{0^n1^m|n,m\geq 1\} \\
	L_9 &=\{0^n1^n0^n|n\geq 1\} \\
	L_{10} &=\{0^n1^m0^k|n,m,k\geq 1\} \\
	L_{11}  &=\{x|x\in\Sigma^{+}\text{且$x$中$0$和$1$的个数相同}\}
	\end{align*}
	\begin{itemize}
		\item 上述所有语言都是$L_4$的子集(子语言);
		\item $L_1,L_2$是有穷语言;其他为无穷语言;其中$L_1$是$\Sigma$上的所有长度为$1$的字组成的语言,$L_2$是$\Sigma$上的所有长度为$2$的字组成的语言;
		\item $L_3,L_4$分别是$\Sigma$的正闭包和克林闭包;
		\item $L_5L_7\ne L_6$,但$L_5L_7=L_8$;同样$L_9\ne L_{10}$,但我们有$L_6\subset L_5L_7, L_9\subset L_{10}$.
		\item $L_6$中的word中的0和1的个数是相同的,并且所有的0在所有的1的前面;$L_{11}$中的word中虽然保持着0和1的个数相同,但它并没有要求所有的0在所有的1的前面。例如,$0101,1100\in L_{11}$,但是$0101\notin L_6$。而对$\forall x\in L_6$,有$x \in L_{11}$。所以$L_6\subset L_{11}$。
	\end{itemize}
\end{example}

\begin{example} $x^T example$
	\begin{enumerate}
		\item $\{x|x=x^T,x\in\Sigma\}$
		\item $\{xx^T|x\in\Sigma^{+}\}$
		\item $\{xx^T|x\in\Sigma^{\ast}\}$
		\item $\{xwx^T|x,w\in\Sigma^{+}\}$
		\item $\{xx^Tw|x,w\in\Sigma^{+}\}$
	\end{enumerate}
\end{example}

\begin{itemize}
	\item 幂
	$\forall L\in\Sigma^{\ast},L$的$n$次\emph{幂}是一个语言,该语言定义为
	\begin{enumerate}
		\item 当$n=0$时,$L^n=\{\epsilon\}$
		\item 当$n\ge 1$时,$L^n=L^{n-1}L$
	\end{enumerate}
	\item 正闭包
	\[L^+=L\cup L^2\cup L^3\cup L^4 \cup\dots\]
	\item 克林闭包
	\[L^{\ast}=L^0\cup L\cup L^2\cup L^3\cup L^4 \cup\dots\]
\end{itemize}

\section{Exercise and Solution}

\begin{exercise}\label{Rl_def}
	设$L$是$\Sigma$上的一个语言,$\Sigma^\ast$上的二元关系$R_L$定义为: 对任给的$x,y\in\Sigma^\ast$,如果对于$\forall z\in\Sigma^\ast$,均有$xz\in L$与$yz\in L$同时成立或者同时不成立,则$xR_L y$。请证明$R_L$是$\Sigma^\ast$上的一个等价关系。将$R_L$称为由语言$L$所确定的等价关系。即
	$$xR_Ly\Leftrightarrow(\forall z\in\Sigma^\ast,xz\in L\Leftrightarrow yz\in L)$$
	
	实际上,$R_L$还有另外一个性质:如果对任给的$x,y\in\Sigma^\ast$, 当$xR_Ly$成立时必有$xzR_Lyz$, 对$\forall z\in\Sigma^\ast$都成立。这将被称为$R_L$的“右不变”性,你能证明此性质成立吗?
\end{exercise}

\begin{solution} 分两步证明$R_L$是“右不变”的等价关系。
	\begin{proof}
		\hfill
		\begin{enumerate}
			\item $R_L$是等价关系。
			\begin{itemize}
				\item 自反性:$\forall x\in\Sigma^\ast$, 显然对于$\forall z\in\Sigma^\ast,xz$要么是$L$的字符串, 要么不是$L$的字符串。由$R_L$的定义知,$xR_Lx$
			    \item 对称性:不难看出,$xR_Ly\Leftrightarrow(\forall z\in\Sigma^\ast,xz\in L\Leftrightarrow yz\in L)\Leftrightarrow yR_Lx$
			    \item 传递性:设$xR_Ly,yR_Lz$。\\
			    $\because$ (由$R_L$的定义知)\\
			    $xR_Ly\Leftrightarrow(\forall w\in\Sigma^\ast, xw\in L\Leftrightarrow yw\in L)$\\
			    $yR_Lz\Leftrightarrow(\forall w\in\Sigma^\ast, yw\in L\Leftrightarrow zw\in L)$\\
			    $\therefore$\\
			    $\forall w\in\Sigma^\ast, xw\in L\Leftrightarrow yw\in L$ and $yw\in L\Leftrightarrow zw\in L$\\
			    $\Rightarrow$\\
			    $\forall w\in\Sigma^\ast, xw\in L\Leftrightarrow zw\in L$
		    \end{itemize}
	    	故\\
	    	$xR_Lz$, 即$R_L$是等价关系。
	    	\item $R_L$是右不变的。\\
	    	设$xR_Ly$。由$R_L$的定义,对$\forall w,v\in\Sigma^\ast,xwv\in L\Leftrightarrow ywv\in L$。注意到$v$的任意性,知$xwR_Lyw$\\
	    	所以,$R_L$是右不变的等价关系。
		\end{enumerate}
	\end{proof}
\end{solution}

\begin{exercise}
	设$\{0,1\}^\ast$上的语言$L=\{0^n1^n|n\ge 0\}$, 请给出$\{0,1\}^\ast$的关于$L$所确定的等价关系$R_L$的等价分类。
\end{exercise}

\begin{solution}
	根据\textbf{Exercise \ref{Rl_def}}对$R_L$的定义及其证明,考虑$R_L$对$\{0,1\}^\ast$的等价分类时,主要需根据语言$L$的结构,分析哪些串按照$L$的要求具有相同的特征。
	\begin{enumerate}
		\item 取$0^n1^n,0^m1^m\in L,0^n1^n\epsilon\in L,0^m1^m\epsilon\in L$同时成立,但是,对所有$x\in\{0,1\}^+,0^n1^nx\in L,0^m1^mx\in L$同时不成立,符合$R_L$的定义。所以,$L$中的元素是属于同一类的。即$0^n1^nR_L0^m1^m\Leftrightarrow(0^n1^n\in L\Leftrightarrow 0^m1^m\in L),n,m\ge 0$。		
		
		\item 分析是否还有其他的元素与$L$中的元素属于同一类。根据$L$的结构,一种类型的串不含子串$10$,这种串可以表示成$0^k1^h,k\ne h$; 另一种是含子串$10$($L$的字符串不含这种子串),这种串可以表示成$x10y$。显然,$01\epsilon\in L$,但是$0^k1^h\notin L(k\ne h),x10y\notin L(x,y\in\{0,1\}^\ast)$。根据等价分类的性质,$\{0,1\}^\ast$中的不在$L$中的串与在$L$的串不在同一个等价类中。
		
		\item 考察不含子串10的串。这些串有如下几种形式:
		\begin{enumerate}
			\item $0^n,n\ge 1$。
			\item $1^n,n\ge 1$。
			\item $0^m1^n,m,n\ge 1,\text{且}m>n$。
			\item $0^m1^n,m,n\ge 1,\text{且}m<n$。
		\end{enumerate}
		对于$0^n,n\ge 1$,$1^n$接在它后面时,构成串$0^n1^n$。显然,当$m\ne n$时,$0^n1^n\in L$, 但$0^m1^n\notin L$。所以,$0^n$和$0^m$一定不在同一个等价类中。
		
		类似的讨论可知,对于$0^n,n\ge 1$:\\	
		$0^n$不可能与形如$0^m1^n(m,n\ge 1,\text{且}m<n)$的串在同一等价类中;\\
		$0^n$不可能与含有子串$10$的串在同一等价类中。
		
		下面再考虑形如$0^k$的串和形如$0^m1^n(m,n\ge 1,\text{且}m>n)$是否可能在同一等价类中。
		
		注意到当$m-n=h$时,
		$$0^h1^h\in L,0^m1^n1^h\in L$$
		同时成立,但是当$n\ge 1,x=01^{h+1}$时$(x\ne 1^h)$,
		$$0^hx\in L,0^m1^nx\notin L$$
		成立。所以,对应$h>0$,令
		$$[h]=\{0^m1^n|m-n=h\text{且}n\ge 1\}$$
		[h]中的元素在同一个等价类中,而且所有其他的元素都不在这个等价类中。实际上,当$h=0$时,有
		$$[0] = L$$
		
		\item 形如$1^m$的串和形如$0^m1^n(m,n\ge 1\text{且}m<n)$的串应该在同一等价类中。	事实上,对于$\{0,1\}^\ast$中的任意字符串$x$,
		$$1^mx\notin L,0^m1^nx\notin L(m,n\ge 1\text{且}m<n)$$
		恒成立。所以,这些字符串在同一个等价类中。
		
		\item 所有含子串$10$的串在同一等价类中。事实上,设$y,z$是含子串$10$的串,对于$\{0,1\}^\ast$中的任意字符串$x$,
			$$yx\notin L,zx\notin L(m,n\ge 1\text{且}m<n)$$
		恒成立。所以,这些字符串在同一个等价类中。
		
		\item 形如$1^m$的串和含子串$10$的串应该在同一等价类中。	事实上,设$y$是含有子串$10$的串,对于$\{0,1\}^\ast$中的任意字符串$x$,
		$$1^mx\notin L,yx\notin L(m\ge 1)$$
		恒成立。所以,这些字符串在同一个等价类中。
		\end{enumerate}
	
		综上所述,$R_L$确定的$\{0,1\}^\ast$的等价分类为\\
		$[10] =\{x10y|x,y\in\{0,1\}^*\}\cup\{0^m1^n|n-m\ge 1\}$ \\
		$[0] =\{0^m1^n|n-m=0\}=\{0^n1^n|n\ge 0\}$ \\
		$[1] =\{0^m1^n|n-m=1\}$ \\
		$[2] =\{0^m1^n|n-m=2\}$ \\
		$\vdots$ \\
		$[h] =\{0^m1^n|n-m=h\}$ \\
		$\vdots$ \\
		$\{0\}$ \\
		$\{00\}$ \\
		$\vdots$ \\
		$\{0^n\}$ \\
		$\vdots$ \\
		其中,$n,m$均为非负整数。
\end{solution}

\begin{exercise} 使用归纳法证明:
	
	对字母表$\Sigma$的任意字符串$x,x$的前缀有$|x|+1$个。
\end{exercise}

\begin{solution}
	\begin{proof}
		设$x\in\Sigma^\ast$,现对$x$的长度施归纳。为了叙述方便用$prefix(x)$表示字符串$x$的所有前缀组成的集合。
		
		当$|x|=0$时,有$x=\epsilon$,由字符串的前缀定义知,$prefix(x)=\{x\}$。\\
		$\epsilon$就是$x$的唯一前缀,而
		\begin{align*}
		|prefix(x) &=|\{\epsilon\}| \\
		           &=1  \\
		           &=0+1 \\
		           &=|x|+1 \\
		\end{align*}
		所以,结论对$|x|=0$成立。
		
		设$|x|=n$时结论成立,$n\ge 0$。即

		$|prefix(x)=|x|+1$\\ 
		现在考察$|x|=n+1$的情况。为了叙述方便,不妨设$x=ya$,其中$|y|=n,a\in\Sigma$。由归纳假设,
		
		$|prefix(y)=|y|+1$ 
		
		首先证明$y$的任何前缀都是$x$的前缀。事实上,设
		
		$|prefix(y)|=\{u_1,u_2,\cdots,u_n\}$	
		
		对于$\forall u\in prefix(y)$,根据前缀的定义,存在$v\in\Sigma^\ast$,使得$uv=y$,注意到$uva=x$,所以,$u$也是$x$的前缀,它对应$x$的后缀为$va$。
		
		再注意到$x=ya$,所以,一方面,对于$\forall u\in prefix(x)$,均有
		
		$u\ne x$\\
		从而
		
		$x\notin prefix(y)$,\\
		然而,由
		
		$x\epsilon=x$\\
		可知,$x$是$x$的一个前缀。另一方面,由$x=ya$知道,如果$u$是$x$的一个前缀,$v$是$u$对应的$x$的后缀,则有如下两种情况:
		\subitem(1) $v\ge 1$,此时必有$u\in prefix(y)$。
		\subitem(2) $|v|=0$,此时必有$v=\epsilon$并且$u=u\epsilon=ya=x$。
		
		\noindent 由此可见,
		\begin{align*}
			prefix(x)&=prefix(y)\cup\{x\} \\
			&=\{u_1,u_2,\cdots,u_n,x\}
		\end{align*}
		由$x\notin prefix(y)$可知,
		\begin{align*}
		|prefix(x)| &= |prefix(y)\cup\{x\}| \\
		&=|prefix(y)|+|\{x\}| \\
		&=|prefix(y)|+1
		\end{align*}
		再由归纳假设,	
		\begin{align*}
			|prefix(x)| &= |prefix(y)|+1 \\
			&=|y|+1+1 \\
			&=|x|+1
		\end{align*}
		
		表明结论对$|x|=n+1$成立。
		
		由归纳法原理,结论对于任意$x\in\Sigma^\ast$成立。$\hfill\square$	
	\end{proof}
    
    \paragraph{另一种往证|x|=n+1时结论成立方法:}
    设$|x|=n$时结论成立,$n\ge 0$。即$|prefix(x)=|x|+1$
     
    现在考察$|x|=n+1$的情况。为了叙述方便,不妨设$x=ya$,其中$|y|=n,a\in\Sigma\& a\ne y$。
    
    $x=ya\Rightarrow$
    \begin{align*}
      prefix(x)&=prefix(ya) &\\
               &=\{\epsilon\}\cup prefix(y) \cup prefix(\{ya\}) &\text{(前缀定义)}\\
               &=\{\epsilon\}\cup prefix(y) \cup prefix(a) & (prefix(\{ya\})=\{\epsilon\}\cup prefix(y)\cup prefix(a)) \\
               &=\{\epsilon\}\cup prefix(y) \cup \{a\} &  (prefix(a)=\{\epsilon,a\})
    \end{align*}
    $\because \epsilon\in prefix(y),a\ne y$,\\
    $\therefore |prefix(x)|= |prefix(y)| + |\{a\}| = |prefix(y)| + 1$\\
    由归纳假设有,$|prefix(y)|=|y|+1$\\
    所以,$|prefix(x)|= |prefix(y)| + 1 = (|y|+1)+1 = |x|+1$\\
    表明结论对$|x|=n+1$成立。
\end{solution}

\begin{exercise}
	设$\Sigma=\{a,b\}$,求字符串$aaaaabbbba$的所有前缀的集合,后缀的集合,真前缀的集合,真后缀的集合。
\end{exercise}

\begin{solution}
	见表(\ref{tab:prepost})。
	
	从表中可以看出,按照前缀与后缀的对应关系构成原来的字符串的对应关系,前缀和后缀是一一对应的,但是,按照这种对应关系,真前缀和真后缀不是一一对应的。不过,若将真后缀和真前缀中的空串$\epsilon$去掉,则这种一一对应关系是仍然存在的。
	
	另外,从对这一问题的讨论知,并不是所有的字符串都有真前缀和真后缀。
	\begin{table}[htbp]
	\caption{$\Sigma=\{a,b\},aaaaabbbba$的前缀,后缀,真前缀,真后缀的集合}
	\label{tab:prepost}
	\begin{tabular}{|c|c|c|c|c|}
		\hline 
		前缀长度&前缀   &是真前缀  &对应后缀      &是真后缀  \\ 
		\hline 
		0&$\epsilon$   &$\surd$  &$aaaaabbbba$ &         \\ 
		\hline 
		1&$a$          &$\surd$  &$aaaabbbba$  &$\surd$  \\ 
		\hline 
		2&$aa$         &$\surd$  &$aaabbbba$   &$\surd$  \\ 
		\hline
	    3&$aaa$        &$\surd$  &$aabbbba$    &$\surd$  \\ 
		\hline
		4&$aaaa$       &$\surd$  &$abbbba$     &$\surd$  \\ 
		\hline
		5&$aaaaa$      &$\surd$  &$bbbba$      &$\surd$  \\
		\hline
		6&$aaaaab$     &$\surd$  &$bbba$       &$\surd$  \\ 
		\hline  
		7&$aaaaabb$    &$\surd$  &$bba$        &$\surd$  \\ 
		\hline
		8&$aaaaabbb$   &$\surd$  &$ba$         &$\surd$  \\ 
		\hline
		9&$aaaaabbbb$  &$\surd$  &$a$          &$\surd$  \\ 
		\hline
		10&$aaaaabbbba$&         &$\epsilon$   &$\surd$  \\ 
		\hline  
	\end{tabular} 
    \end{table}
\end{solution}

\begin{exercise}
	设$L_1,L_2,L_3,L_4$分别是$\Sigma_1,\Sigma_2,\Sigma_3,\Sigma_4$上的语言,能否说$L_1,L_2,L_3,L_4$是某个字母表$\Sigma$上的语言?如果能,请问这个字母表$\Sigma$是什么样的?。
\end{exercise}

\begin{solution}
	可以说$L_1,L_2,L_3,L_4$是同一个字母表$\Sigma$上的语言。这里
	$$\Sigma=\Sigma_1\cup\Sigma_2\cup\Sigma_3\cup\Sigma_4$$
\end{solution}

\begin{exercise}
	设$L_1,L_2,L_3,L_4$分别是$\Sigma_1,\Sigma_2,\Sigma_3,\Sigma_4$上的语言,证明下列等式成立。
	$$(L_1\cup L_2\cup L_3\cup L_4)^\ast=(L_1^\ast L_2^\ast L_3^\ast L_4^\ast)^\ast$$
\end{exercise}

\begin{solution}
	\begin{proof}
		考虑到语言就是一系列字符串的集合,所以,证明两个语言相等,实际上就是证明相应的两个集合相等。因此,为证明
		$$(L_1\cup L_2\cup L_3\cup L_4)^\ast=(L_1^\ast L_2^\ast L_3^\ast L_4^\ast)^\ast$$
		就是证明下列(\ref{eq:1}),(\ref{eq:2})式同时成立。
		\begin{align*}
		(L_1\cup L_2\cup L_3\cup L_4)^\ast &\subseteq(L_1^\ast L_2^\ast L_3^\ast L_4^\ast)^\ast \tag{1}\label{eq:1} \\
		(L_1\cup L_2\cup L_3\cup L_4)^\ast &\supseteq(L_1^\ast L_2^\ast L_3^\ast L_4^\ast)^\ast \tag{2}\label{eq:2}
		\end{align*}
		首先证明(\ref{eq:1})式成立。为此,设
		
		$x\in (L_1\cup L_2\cup L_3\cup L_4)^\ast$\\
		从而存在非负整数$n$和$x_1,x_2,\cdots,x_n,\{x_1,x_2,\cdots,x_n\}\subseteq(L_1\cup L_2\cup L_3\cup L_4)$, 使得
		
		$x=x_1x_2\cdots x_n$\\
		注意到$\{x_1,x_2,\cdots,x_n\}\subseteq(L_1\cup L_2\cup L_3\cup L_4)$,所以,对于$1\le j\le n$
		
		$x_j\in L_1,x_j\in L_2,x_j\in L_3,x_j\in L_4$\\
		中至少一个成立,这表明
		
		$x_j\in L_1^\ast,x_j\in L_2^\ast,x_j\in L_3^\ast,x_j\in L_4^\ast$\\
		中至少一个成立,再注意到
		
		$\epsilon\in L_1^\ast,\epsilon\in L_2^\ast,\epsilon\in L_3^\ast,\epsilon\in L_4^\ast$\\
		并且
		
		$L_1\subseteq L_1^\ast,L_2\subseteq L_2^\ast,L_3\subseteq L_3^\ast,L_4\subseteq L_4^\ast$\\
		使得
		
		$L_1\subseteq L_1^\ast L_2^\ast L_3^\ast L_4^\ast,L_2\subseteq L_1^\ast L_2^\ast L_3^\ast L_4^\ast,L_3\subseteq L_1^\ast L_2^\ast L_3^\ast L_4^\ast,L_4\subseteq L_1^\ast L_2^\ast L_3^\ast L_4^\ast$\\
		从而
		
		$x_j\in  L_1^\ast L_2^\ast L_3^\ast L_4^\ast$\\
		故
		
		$x=x_1x_2\cdots x_n \in  (L_1^\ast L_2^\ast L_3^\ast L_4^\ast)^n$\\
		亦即
		
		$x=x_1x_2\cdots x_n \in  (L_1^\ast L_2^\ast L_3^\ast L_4^\ast)^\ast$\\
		所以(\ref{eq:1})式成立。\\
		类似易证(\ref{eq:2})式亦成立。
		
		综上所述(\ref{eq:1}),(\ref{eq:2})式同时成立。所以,$(L_1\cup L_2\cup L_3\cup L_4)^\ast=(L_1^\ast L_2^\ast L_3^\ast L_4^\ast)^\ast$。$\hfill\square$	
	\end{proof}
\end{solution}

\begin{exercise}
	设$L_1,L_2,L_3,L_4$分别是$\Sigma_1,\Sigma_2,\Sigma_3,\Sigma_4$上的语言,证明下列等式成立否。
	$$L_2(L_1L_2\cup L_2)^\ast L_1=L_1L_1^\ast L_2(L_1L_1^\ast L_2)^\ast$$
\end{exercise}

\begin{solution}
	\begin{proof}
		此式不成立,仅举一个反例即可完成证明。
		
		令 $L_1=\{a\},L_2=\{b\}$,此时,
		
		$L_2(L_1L_2\cup L_2)^\ast L_1=\{b\}(\{a\}\{b\}\cup\{b\})^\ast\{a\}=\{b\}\{ab,b\}^\ast\{a\}$\\
		它含的串都是以$\{b\}开头,以\{a\}结尾的串$。
		
		$L_1L_1^\ast L_2(L_1L_1^\ast L_2)^\ast=\{a\}\{a\}^\ast\{b\}(\{a\}\{a\}^\ast\{b\})^\ast=\{a\}^+\{b\}(\{a\}^+\{b\})^\ast$\\
		它含的串都都是以$\{a\}$开头的串。
		
		所以,此式不成立。$\hfill\square$
	\end{proof}
\end{solution}

\begin{exercise}
	设$\Sigma=\{0,1\}$请给出$\Sigma$上的下列语言的形式化表示。
	\begin{enumerate}
		\item 所有长度为偶数的串。
		\item 所有含有3个连续0的串。
		\item 所有的倒数第10个字符是0的串。
	\end{enumerate}
\end{exercise}

\begin{solution} $\Sigma=\{0,1\}$
	\begin{enumerate}
		\item 所有长度为偶数的串。可以用以下任一种表示方法(含$\epsilon,|\epsilon|=0$,认为是偶数):
		\begin{enumerate}
			\item $(\{0,1\}\{0,1\})^\ast$
			\item $(\{00,01,10,11\})^\ast$
			\item $\{00,01,10,11\}\cup (\{0,1\}\{0,1\})^\ast$
		\end{enumerate}
		\item 所有含有3个连续0的串。\\
		$\{0,1\}^\ast 000\{0,1\}^\ast$
		\item 所有的倒数第10个字符是0的串。\\
		$\{0,1\}^\ast0\{0,1\}\{0,1\}\{0,1\}\{0,1\}\{0,1\}\{0,1\}\{0,1\}\{0,1\}\{0,1\}$
	\end{enumerate}
\end{solution}
